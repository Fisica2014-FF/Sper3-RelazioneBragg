\documentclass[11pt,a4paper]{article} % Prepara un documento con un font grande

\usepackage{iftex}

\ifLuaTeX
  % Adatta LaTeX alle convenzioni tipografiche italiane,
% e ridefinisce alcuni titoli in italiano, come "Capitolo" al posto di "Chapter",
% se il documento è in italiano
%\usepackage[italian]{babel}
%\usepackage[utf8]{inputenc} % Consente l'uso caratteri accentati italiani
%\usepackage{graphicx}		% Per le immagini
%\usepackage{gnuplot-lua-tikz}
%\usepackage[top=2.5cm, bottom=2cm, left=2cm, right=2cm]{geometry}

%\nonstopmode %non fermarti agli errori

%\usepackage{fancyhdr}
%\setlength{\headheight}{15.2pt}
%\pagestyle{fancy} % Solo le pagine normali, non i titoli nè la pagina iniziale


%%%%%%%%%%%%%%%%%%%%%%%%%%%%%%%%%%%%%%%%%%%%%%%%%%%%%%%%%%%%%%%%%%%%%%%%%%%%%%%%%%%%%%%%%

\usepackage{lipsum} % Package to generate dummy text throughout this template

\usepackage{fontspec}
\setmainfont[Ligatures=TeX]{Alegreya}

%\usepackage[sc]{mathpazo} % Use the Palatino font
%\usepackage[T1]{fontenc} % Use 8-bit encoding that has 256 glyphs
%%%%%
%\usepackage{Alegreya} %% Option 'black' gives heavier bold face 
%\renewcommand*\oldstylenums[1]{{\AlegreyaOsF #1}}

%\usepackage[euler-digits,euler-hat-accent]{eulervm}
%%%%%%
%\usepackage[utf8]{inputenc} % Consente l'uso caratteri accentati italiani
%\linespread{1.05} % Line spacing - Palatino needs more space between lines
\usepackage{amsmath, amsthm, amssymb, amsfonts}
\usepackage{microtype} % Slightly tweak font spacing for aesthetics

%%%%%%%%%%%%%%%%%%%%%%%%%%%%%%%%%%%%%%%%%%%%%
%Miei package
\usepackage[italian]{babel}
\usepackage{graphicx}		% Per le immagini
\usepackage{gnuplot-lua-tikz}
%%%%%%%%%%%%%%%%%%%%%%%%%%%%%%%%%%%%%%%%%%%%%
\usepackage[hmarginratio=1:1,top=32mm,columnsep=20pt]{geometry} % Document margins
\usepackage{multicol} % Used for the two-column layout of the document
\usepackage[hang, small,labelfont=bf,up,textfont=it,up]{caption} % Custom captions under/above floats in tables or figures
\usepackage{booktabs} % Horizontal rules in tables
\usepackage{float} % Required for tables and figures in the multi-column environment - they need to be placed in specific locations with the [H] (e.g. \begin{table}[H])
\usepackage{hyperref} % For hyperlinks in the PDF

\usepackage{lettrine} % The lettrine is the first enlarged letter at the beginning of the text
\usepackage{paralist} % Used for the compactitem environment which makes bullet points with less space between them

\usepackage{abstract} % Allows abstract customization
\renewcommand{\abstractnamefont}{\normalfont\bfseries} % Set the "Abstract" text to bold
\renewcommand{\abstracttextfont}{\normalfont\small\itshape} % Set the abstract itself to small italic text

\usepackage{titlesec} % Allows customization of titles
\renewcommand\thesection{\Roman{section}} % Roman numerals for the sections
\renewcommand\thesubsection{\Roman{subsection}} % Roman numerals for subsections
\titleformat{\section}[block]{\large\scshape\centering}{\thesection.}{1em}{} % Change the look of the section titles
\titleformat{\subsection}[block]{\large}{\thesubsection.}{1em}{} % Change the look of the section titles

\usepackage{fancyhdr} % Headers and footers
\pagestyle{fancy} % All pages have headers and footers
\fancyhead{} % Blank out the default header
\fancyfoot{} % Blank out the default footer
\fancyhead[C]{Running title $\bullet$ November 2012 $\bullet$ Vol. XXI, No. 1} % Custom header text
\fancyfoot[RO,LE]{\thepage} % Custom footer text


\else
  \input{./preamboli_e_stili/pacchetti.tex}
\fi
\DeclareGraphicsExtensions{.pdf, .png, .jpg} % Se due immagini hanno lo stesso nome sceglile secondo l'ordine di filetype qui
\graphicspath{ {./img/} }					 % Path delle immagini 

\title{
\vspace{-3cm}
\fontsize{48pt}{10pt}\selectfont
\textsc{Relazione di \\[3mm] Spettroscopia} \\[8mm] 
\fontsize{24pt}{10pt}\selectfont
\textit{Camera di Bragg}
}
% \title{}
\author{
\large
\textsc{Francesco Forcher}\\[2mm]
\normalsize Università di Padova, Facoltà di Fisica\\
\normalsize \texttt{francesco.forcher@studenti.unipd.it}\\
\normalsize Matricola: \texttt{1073458}\\
\and
\large
\textsc{Enrico Lusiani}\\[2mm]
\normalsize Università di Padova, Facoltà di Fisica\\
\normalsize \texttt{enrico.lusiani@studenti.unipd.it}\\
\normalsize Matricola: \texttt{1073300}\\
\and
\large
\textsc{Laura Buonincontri}\\[2mm]
\normalsize Università di Padova, Facoltà di Fisica\\
\normalsize \texttt{laura.buonincontri@studenti.unipd.it}\\
\normalsize Matricola: \texttt{1073131}
}
\date{\today}

\input{./preamboli_e_stili/stili_float.tex}


%////////////////////////////////////////////////////////////////////////////////////////////////////////////////////////////
%////////////////////////////////////////////////////////////////////////////////////////////////////////////////////////////
% Fine dei dati iniziali per il latex: il documento finale inizierà da qui
\begin{document}

%\includepdf[pages={1}]{./img/copertina_relazione.pdf}

{
%\color{grigio-molto-scuro}
%\lsstyle % Abilita il letterspacing personalizzato
%\unclfamily % Cagata per il font Uncial

\maketitle % Produce il titolo a partire dai comandi \title, \author e \date

\vspace{ \stretch{1} }
\begin{center}
	%non si poteva non mettere
	\includegraphics[width=0.7\textwidth]{stelle_di_natale_quadrimensionale.png}
	%\caption{Una NTupla in 4D} 
\end{center}

%\vspace{ \stretch{1} }
\newpage
% Le varie sezioni
%\section{Obiettivi}
\begin{abstract}
	\noindent
	Studio della risposta alla radiazione alfa di una camera di Bragg
Analisi dei segnali per calcolare l'energia depositata e l'altezza del picco di Bragg
Studio della risposta al variare dell'energia della particella e al variare della pressione
Misurazione del range delle particelle e verifica della relazione tra range e pressione

\end{abstract}

\newpage


\microtypesetup{protrusion=false} % disables protrusion locally in the document
\tableofcontents % prints Table of Contents
%\listoftabella
%\listoffigures
%\listoftables
\microtypesetup{protrusion=true} % enables protrusion

%\begin{multicols}{2}

\section{Schema camera}
	\begin{center}
\includegraphics[width=0.7\textwidth]{apparato.png}
\end{center}

% \section{Metodologia di misura}
% 	\input{./sezioni/metodologia_misura.tex}

%\newpage
%\end{multicols}
% \section{Presentazione dei dati}
 	%inserire circuito
\subsection{Amplificatore invertente}
Schema amplificatore invertente:
%inserire circuito
Le resistenze sono state scelte in modo da avere guadagno $A=-10 \frac{V}{V}$\\
$R_1=9.85 \pm 0.05\,k\Omega $\\ %mettere errore
$R_2=101.3 \pm 0.6\,k\Omega$\\ %mettere errore
$R_3=56.0 \pm 0.3\,\Omega$\\ %mettere errore

Per il calcolo degli errori sul valore delle resistenze, lette sull'Agilent U1232A, è stata utilizzata la seguente formula:
$$\sigma_\textrm{tot}=\sqrt{ \sigma^{2} _\textrm{\%} + \sigma^{2} _\textrm{dgt}}$$
Per il calcolo delle $\sigma_\textrm{tot}$ è stato cercato del datasheet dello strumento, l'errore percentuale e di digit
corrispondente al fondo scala utilizzato.

\subsubsection{Calcolo amplificazione}
La relazione tra le resistenze, affichè soddisfino la richiesta A=10 è la seguente:

$$\frac{V_1-V_n}{R_1}=\frac{V_n-V_0}{R_2}$$
$$V_n=0$$
$$\frac{V_1}{R_1}=\frac{-V_0}{R_2}$$
$$V_0=-\frac{R_2}{R_1} \cdot V_1$$
Da cui si ricava la relazione per il calcolo di A.

\subsubsection{Analisi}
La stima di A teorica, a partire dalle resistenze misurate è:\\
$A_\textrm{teorica}=10.28 \pm 0.08$ %inserire valore teorico

Le misure sono state fatte applicando una tensione sinusoidale di frequenza $ f=1 \,kHz$, variando l'ampiezza tra 
$0.2 V_\textrm{pp}$ e $4 V_\textrm{pp}$.

In seguito è stato fatto il grafico della curva di trasferimento di un amplificatore invertente.

\begin{grafico}
 \centering
 \resizebox{\textwidth}{!}{%
 \input{../grafici/risultati/amp_inv.tex}
 }%
 \caption{Curva di trasferimento di un amplificatore invertente} 
 \label{gr:amp_inv.tex} 
\end{grafico}


In seguito sono presentati i dati del  \autoref{gr:amp_inv.tex} acquisiti in laboratorio, con i rispettivi errori:
\begin{tabella}
 \centering
 \input{tabelle/tab_inv.tex} 
 \caption{Dati curva di trasferimento}
 \label{tab:tab_inv.tex}
\end{tabella}

Per il calcolo degli errori sui valori di $V_\textrm{in}$ e $V_\textrm{out}$ letti sull'oscilloscopio, è stata utilizzata la seguente formula:
$$\sigma_\textrm{tot}=\sqrt{ (0.02\cdot V_\textrm{letto})^2 + (0.06 \cdot V_\textrm{div})^2}$$

E' stata fatta l'interpolazione lineare dei punti nel \autoref{gr:amp_inv.tex} pesata dei punti compresi tra 0 e 1.5 V.\\
$q = 0.02 \pm 0.03 \,V$\\
$m = -10.0 \pm 0.1 \, \frac{V}{V}$




\subsection{Amplificatore non invertente}

Schema amplificatore non invertente:
%inserire circuito
Le resistenze sono state scelte in modo da avere guadagno $A=10 \frac{V}{V}$\\
$R_\textrm{1,up}=9.91 \pm0.05 \,k\Omega $\\ %mettere errore
$R_\textrm{1,down}=9.85 \pm 0.05\,k\Omega$\\ %mettere errore
$R_\textrm{2,up}=99.7 \pm 0.6\,k\Omega$\\ %mettere errore
$R_\textrm{2,down}=101.3 \pm 0.6\,k\Omega$\\
$R_4=56.0 \pm 0.3\,\Omega$

\subsubsection{Calcolo amplificazione}
La relazione tra le resistenze, affichè soddisfino la richiesta A=10 è la seguente:
Nell'ingresso non invertente:
$$\frac{V_1-V_p}{R_\textrm{1down}}=\frac{V_p}{R_\textrm{2down}}$$
$$\frac{V_1}{R_\textrm{1down}}=\frac{V_p}{R_\textrm{1down}}+\frac{V_p}{R_\textrm{2down}}=V_p \left(\frac{1}{R_\textrm{1down}}+\frac{1}{R_\textrm{2down}}\right)$$

Nell'ingresso invertente:
$$\frac{V_0-V_n}{R_\textrm{2up}}=\frac{V_n}{R_\textrm{1up}}$$
$$\frac{V_0}{R_\textrm{2up}}=\frac{V_n}{R_\textrm{1up}}+\frac{V_n}{R_\textrm{2up}}=V_n \left(\frac{1}{R_\textrm{1up}}+\frac{1}{R_\textrm{2up}}\right)$$

Poichè 
$$V_p=V_n$$

$$\frac{V_1}{R_\textrm{1down}} \frac{1}{\left(\frac{1}{R_\textrm{1down}}+\frac{1}{R_\textrm{2down} }\right)}=\frac{V_0}{R_\textrm{2up}} \frac{1}{\left(\frac{1}{R_\textrm{1up}}+\frac{1}{R_\textrm{2up}} \right)}$$
$$V_0=\frac{R_\textrm{2up}}{R_\textrm{1down}} \cdot V_1 \frac{\left(\frac{1}{R_\textrm{1up}}+\frac{1}{R_\textrm{2up} }\right)}{\left(\frac{1}{R_\textrm{1down}}+\frac{1}{R_\textrm{2down}} \right)}$$
Da cui si ricava la relazione per il calcolo di A.
Se poi si assume che $R_\textrm{1down}=R_\textrm{1up}$ e $R_\textrm{2down}=R_\textrm{2up}$, la relazione si semplifica a 
$$V_0=\frac{R_\textrm{2up}}{R_\textrm{1down}} \cdot V_1$$
\subsubsection{Analisi}
La stima di A teorica, a partire dalle resistenze misurate è:\\
$A_\textrm{teorica}=10.08 \pm 0.07$ %inserire valore teorico

Le misure sono state fatte applicando una tensione sinusoidale di frequenza $ f=1 \,kHz$, variando l'ampiezza tra 
$0.2 V_\textrm{pp}$ e $4 V_\textrm{pp}$.

In seguito è stato fatto il grafico della curva di trasferimento di un amplificatore non invertente.

\begin{grafico} 
 \centering
 \resizebox{\textwidth}{!}{%
 \begin{tikzpicture}
\pgfdeclareplotmark{cross} {
\pgfpathmoveto{\pgfpoint{-0.3\pgfplotmarksize}{\pgfplotmarksize}}
\pgfpathlineto{\pgfpoint{+0.3\pgfplotmarksize}{\pgfplotmarksize}}
\pgfpathlineto{\pgfpoint{+0.3\pgfplotmarksize}{0.3\pgfplotmarksize}}
\pgfpathlineto{\pgfpoint{+1\pgfplotmarksize}{0.3\pgfplotmarksize}}
\pgfpathlineto{\pgfpoint{+1\pgfplotmarksize}{-0.3\pgfplotmarksize}}
\pgfpathlineto{\pgfpoint{+0.3\pgfplotmarksize}{-0.3\pgfplotmarksize}}
\pgfpathlineto{\pgfpoint{+0.3\pgfplotmarksize}{-1.\pgfplotmarksize}}
\pgfpathlineto{\pgfpoint{-0.3\pgfplotmarksize}{-1.\pgfplotmarksize}}
\pgfpathlineto{\pgfpoint{-0.3\pgfplotmarksize}{-0.3\pgfplotmarksize}}
\pgfpathlineto{\pgfpoint{-1.\pgfplotmarksize}{-0.3\pgfplotmarksize}}
\pgfpathlineto{\pgfpoint{-1.\pgfplotmarksize}{0.3\pgfplotmarksize}}
\pgfpathlineto{\pgfpoint{-0.3\pgfplotmarksize}{0.3\pgfplotmarksize}}
\pgfpathclose
\pgfusepathqstroke
}
\pgfdeclareplotmark{cross*} {
\pgfpathmoveto{\pgfpoint{-0.3\pgfplotmarksize}{\pgfplotmarksize}}
\pgfpathlineto{\pgfpoint{+0.3\pgfplotmarksize}{\pgfplotmarksize}}
\pgfpathlineto{\pgfpoint{+0.3\pgfplotmarksize}{0.3\pgfplotmarksize}}
\pgfpathlineto{\pgfpoint{+1\pgfplotmarksize}{0.3\pgfplotmarksize}}
\pgfpathlineto{\pgfpoint{+1\pgfplotmarksize}{-0.3\pgfplotmarksize}}
\pgfpathlineto{\pgfpoint{+0.3\pgfplotmarksize}{-0.3\pgfplotmarksize}}
\pgfpathlineto{\pgfpoint{+0.3\pgfplotmarksize}{-1.\pgfplotmarksize}}
\pgfpathlineto{\pgfpoint{-0.3\pgfplotmarksize}{-1.\pgfplotmarksize}}
\pgfpathlineto{\pgfpoint{-0.3\pgfplotmarksize}{-0.3\pgfplotmarksize}}
\pgfpathlineto{\pgfpoint{-1.\pgfplotmarksize}{-0.3\pgfplotmarksize}}
\pgfpathlineto{\pgfpoint{-1.\pgfplotmarksize}{0.3\pgfplotmarksize}}
\pgfpathlineto{\pgfpoint{-0.3\pgfplotmarksize}{0.3\pgfplotmarksize}}
\pgfpathclose
\pgfusepathqfillstroke
}
\pgfdeclareplotmark{newstar} {
\pgfpathmoveto{\pgfqpoint{0pt}{\pgfplotmarksize}}
\pgfpathlineto{\pgfqpointpolar{44}{0.5\pgfplotmarksize}}
\pgfpathlineto{\pgfqpointpolar{18}{\pgfplotmarksize}}
\pgfpathlineto{\pgfqpointpolar{-20}{0.5\pgfplotmarksize}}
\pgfpathlineto{\pgfqpointpolar{-54}{\pgfplotmarksize}}
\pgfpathlineto{\pgfqpointpolar{-90}{0.5\pgfplotmarksize}}
\pgfpathlineto{\pgfqpointpolar{234}{\pgfplotmarksize}}
\pgfpathlineto{\pgfqpointpolar{198}{0.5\pgfplotmarksize}}
\pgfpathlineto{\pgfqpointpolar{162}{\pgfplotmarksize}}
\pgfpathlineto{\pgfqpointpolar{134}{0.5\pgfplotmarksize}}
\pgfpathclose
\pgfusepathqstroke
}
\pgfdeclareplotmark{newstar*} {
\pgfpathmoveto{\pgfqpoint{0pt}{\pgfplotmarksize}}
\pgfpathlineto{\pgfqpointpolar{44}{0.5\pgfplotmarksize}}
\pgfpathlineto{\pgfqpointpolar{18}{\pgfplotmarksize}}
\pgfpathlineto{\pgfqpointpolar{-20}{0.5\pgfplotmarksize}}
\pgfpathlineto{\pgfqpointpolar{-54}{\pgfplotmarksize}}
\pgfpathlineto{\pgfqpointpolar{-90}{0.5\pgfplotmarksize}}
\pgfpathlineto{\pgfqpointpolar{234}{\pgfplotmarksize}}
\pgfpathlineto{\pgfqpointpolar{198}{0.5\pgfplotmarksize}}
\pgfpathlineto{\pgfqpointpolar{162}{\pgfplotmarksize}}
\pgfpathlineto{\pgfqpointpolar{134}{0.5\pgfplotmarksize}}
\pgfpathclose
\pgfusepathqfillstroke
}
\definecolor{c}{rgb}{1,1,1};
\draw [color=c, fill=c] (0,0) rectangle (20,13.5632);
\draw [color=c, fill=c] (2,1.35632) rectangle (18,12.2069);
\definecolor{c}{rgb}{0,0,0};
\draw [c,line width=0.9] (2,1.35632) -- (2,12.2069) -- (18,12.2069) -- (18,1.35632) -- (2,1.35632);
\definecolor{c}{rgb}{1,1,1};
\draw [color=c, fill=c] (2,1.35632) rectangle (18,12.2069);
\definecolor{c}{rgb}{0,0,0};
\draw [c,line width=0.9] (2,1.35632) -- (2,12.2069) -- (18,12.2069) -- (18,1.35632) -- (2,1.35632);
\draw [c,line width=0.9] (2,1.35632) -- (18,1.35632);
\draw [c,dotted,line width=0.9] (3.7792,12.2069) -- (3.7792,1.35632);
\draw [c,dotted,line width=0.9] (6.85062,12.2069) -- (6.85062,1.35632);
\draw [c,dotted,line width=0.9] (9.92204,12.2069) -- (9.92204,1.35632);
\draw [c,dotted,line width=0.9] (12.9935,12.2069) -- (12.9935,1.35632);
\draw [c,dotted,line width=0.9] (16.0649,12.2069) -- (16.0649,1.35632);
\draw [c,dotted,line width=0.9] (3.7792,12.2069) -- (3.7792,1.35632);
\draw [c,dotted,line width=0.9] (16.0649,12.2069) -- (16.0649,1.35632);
\draw [c,line width=0.9] (2,1.35632) -- (2,12.2069);
\draw [c,dotted,line width=0.9] (18,2.19282) -- (2,2.19282);
\draw [c,dotted,line width=0.9] (18,3.69696) -- (2,3.69696);
\draw [c,dotted,line width=0.9] (18,5.2011) -- (2,5.2011);
\draw [c,dotted,line width=0.9] (18,6.70524) -- (2,6.70524);
\draw [c,dotted,line width=0.9] (18,8.20938) -- (2,8.20938);
\draw [c,dotted,line width=0.9] (18,9.71352) -- (2,9.71352);
\draw [c,dotted,line width=0.9] (18,11.2177) -- (2,11.2177);
\draw [c,dotted,line width=0.9] (18,2.19282) -- (2,2.19282);
\draw [c,dotted,line width=0.9] (18,11.2177) -- (2,11.2177);
\draw [c,line width=0.9] (2,1.35632) -- (18,1.35632);
\draw [anchor= east] (18,0.596782) node[scale=1.08496, color=c, rotate=0]{$V_{out} [V]$};
\draw [c,line width=0.9] (3.7792,1.68184) -- (3.7792,1.35632);
\draw [c,line width=0.9] (4.39348,1.51908) -- (4.39348,1.35632);
\draw [c,line width=0.9] (5.00777,1.51908) -- (5.00777,1.35632);
\draw [c,line width=0.9] (5.62205,1.51908) -- (5.62205,1.35632);
\draw [c,line width=0.9] (6.23634,1.51908) -- (6.23634,1.35632);
\draw [c,line width=0.9] (6.85062,1.68184) -- (6.85062,1.35632);
\draw [c,line width=0.9] (7.46491,1.51908) -- (7.46491,1.35632);
\draw [c,line width=0.9] (8.07919,1.51908) -- (8.07919,1.35632);
\draw [c,line width=0.9] (8.69348,1.51908) -- (8.69348,1.35632);
\draw [c,line width=0.9] (9.30776,1.51908) -- (9.30776,1.35632);
\draw [c,line width=0.9] (9.92204,1.68184) -- (9.92204,1.35632);
\draw [c,line width=0.9] (10.5363,1.51908) -- (10.5363,1.35632);
\draw [c,line width=0.9] (11.1506,1.51908) -- (11.1506,1.35632);
\draw [c,line width=0.9] (11.7649,1.51908) -- (11.7649,1.35632);
\draw [c,line width=0.9] (12.3792,1.51908) -- (12.3792,1.35632);
\draw [c,line width=0.9] (12.9935,1.68184) -- (12.9935,1.35632);
\draw [c,line width=0.9] (13.6078,1.51908) -- (13.6078,1.35632);
\draw [c,line width=0.9] (14.222,1.51908) -- (14.222,1.35632);
\draw [c,line width=0.9] (14.8363,1.51908) -- (14.8363,1.35632);
\draw [c,line width=0.9] (15.4506,1.51908) -- (15.4506,1.35632);
\draw [c,line width=0.9] (16.0649,1.68184) -- (16.0649,1.35632);
\draw [c,line width=0.9] (3.7792,1.68184) -- (3.7792,1.35632);
\draw [c,line width=0.9] (3.16491,1.51908) -- (3.16491,1.35632);
\draw [c,line width=0.9] (2.55063,1.51908) -- (2.55063,1.35632);
\draw [c,line width=0.9] (16.0649,1.68184) -- (16.0649,1.35632);
\draw [c,line width=0.9] (16.6792,1.51908) -- (16.6792,1.35632);
\draw [c,line width=0.9] (17.2935,1.51908) -- (17.2935,1.35632);
\draw [c,line width=0.9] (17.9077,1.51908) -- (17.9077,1.35632);
\draw [anchor=base] (3.7792,0.908736) node[scale=1.08496, color=c, rotate=0]{-2};
\draw [anchor=base] (6.85062,0.908736) node[scale=1.08496, color=c, rotate=0]{-1};
\draw [anchor=base] (9.92204,0.908736) node[scale=1.08496, color=c, rotate=0]{0};
\draw [anchor=base] (12.9935,0.908736) node[scale=1.08496, color=c, rotate=0]{1};
\draw [anchor=base] (16.0649,0.908736) node[scale=1.08496, color=c, rotate=0]{2};
\draw [c,line width=0.9] (2,1.35632) -- (2,12.2069);
\draw [anchor= east] (0.88,12.2069) node[scale=1.08496, color=c, rotate=90]{$V_{in} [V]$};
\draw [c,line width=0.9] (2.48,2.19282) -- (2,2.19282);
\draw [c,line width=0.9] (2.24,2.49365) -- (2,2.49365);
\draw [c,line width=0.9] (2.24,2.79447) -- (2,2.79447);
\draw [c,line width=0.9] (2.24,3.0953) -- (2,3.0953);
\draw [c,line width=0.9] (2.24,3.39613) -- (2,3.39613);
\draw [c,line width=0.9] (2.48,3.69696) -- (2,3.69696);
\draw [c,line width=0.9] (2.24,3.99779) -- (2,3.99779);
\draw [c,line width=0.9] (2.24,4.29861) -- (2,4.29861);
\draw [c,line width=0.9] (2.24,4.59944) -- (2,4.59944);
\draw [c,line width=0.9] (2.24,4.90027) -- (2,4.90027);
\draw [c,line width=0.9] (2.48,5.2011) -- (2,5.2011);
\draw [c,line width=0.9] (2.24,5.50193) -- (2,5.50193);
\draw [c,line width=0.9] (2.24,5.80275) -- (2,5.80275);
\draw [c,line width=0.9] (2.24,6.10358) -- (2,6.10358);
\draw [c,line width=0.9] (2.24,6.40441) -- (2,6.40441);
\draw [c,line width=0.9] (2.48,6.70524) -- (2,6.70524);
\draw [c,line width=0.9] (2.24,7.00607) -- (2,7.00607);
\draw [c,line width=0.9] (2.24,7.30689) -- (2,7.30689);
\draw [c,line width=0.9] (2.24,7.60772) -- (2,7.60772);
\draw [c,line width=0.9] (2.24,7.90855) -- (2,7.90855);
\draw [c,line width=0.9] (2.48,8.20938) -- (2,8.20938);
\draw [c,line width=0.9] (2.24,8.51021) -- (2,8.51021);
\draw [c,line width=0.9] (2.24,8.81104) -- (2,8.81104);
\draw [c,line width=0.9] (2.24,9.11186) -- (2,9.11186);
\draw [c,line width=0.9] (2.24,9.41269) -- (2,9.41269);
\draw [c,line width=0.9] (2.48,9.71352) -- (2,9.71352);
\draw [c,line width=0.9] (2.24,10.0143) -- (2,10.0143);
\draw [c,line width=0.9] (2.24,10.3152) -- (2,10.3152);
\draw [c,line width=0.9] (2.24,10.616) -- (2,10.616);
\draw [c,line width=0.9] (2.24,10.9168) -- (2,10.9168);
\draw [c,line width=0.9] (2.48,11.2177) -- (2,11.2177);
\draw [c,line width=0.9] (2.48,2.19282) -- (2,2.19282);
\draw [c,line width=0.9] (2.24,1.89199) -- (2,1.89199);
\draw [c,line width=0.9] (2.24,1.59116) -- (2,1.59116);
\draw [c,line width=0.9] (2.48,11.2177) -- (2,11.2177);
\draw [c,line width=0.9] (2.24,11.5185) -- (2,11.5185);
\draw [c,line width=0.9] (2.24,11.8193) -- (2,11.8193);
\draw [c,line width=0.9] (2.24,12.1201) -- (2,12.1201);
\draw [anchor= east] (1.9,2.19282) node[scale=1.08496, color=c, rotate=0]{-15};
\draw [anchor= east] (1.9,3.69696) node[scale=1.08496, color=c, rotate=0]{-10};
\draw [anchor= east] (1.9,5.2011) node[scale=1.08496, color=c, rotate=0]{-5};
\draw [anchor= east] (1.9,6.70524) node[scale=1.08496, color=c, rotate=0]{0};
\draw [anchor= east] (1.9,8.20938) node[scale=1.08496, color=c, rotate=0]{5};
\draw [anchor= east] (1.9,9.71352) node[scale=1.08496, color=c, rotate=0]{10};
\draw [anchor= east] (1.9,11.2177) node[scale=1.08496, color=c, rotate=0]{15};
\definecolor{c}{rgb}{0,0,1};
\foreach \P in {(13.2392,9.9241), (6.72776,3.48638), (10.2538,7.02712), (9.5934,6.38335), (11.2489,8.00482), (8.61055,5.4207), (12.2072,8.9434), (7.63691,4.46708), (14.1913,10.8867), (5.68348,2.52373), (15.2049,11.1876), (4.70063,2.37331),
 (16.1877,11.1274), (3.80991,2.43348), (16.4949,11.1876), (3.50277,2.43348)}{\draw[mark options={color=c,fill=c},mark size=1.681682pt,mark=*,mark size=1pt] plot coordinates {\P};}
\draw [c,line width=0.9] (5.36098,2.22861) -- (5.45312,2.319) -- (5.54527,2.4094) -- (5.63741,2.4998) -- (5.72955,2.59019) -- (5.82169,2.68059) -- (5.91384,2.77099) -- (6.00598,2.86138) -- (6.09812,2.95178) -- (6.19027,3.04218) -- (6.28241,3.13257)
 -- (6.37455,3.22297) -- (6.46669,3.31337) -- (6.55884,3.40376) -- (6.65098,3.49416) -- (6.74312,3.58456) -- (6.83526,3.67495) -- (6.92741,3.76535) -- (7.01955,3.85575) -- (7.11169,3.94614) -- (7.20384,4.03654) -- (7.29598,4.12694) --
 (7.38812,4.21733) -- (7.48026,4.30773) -- (7.57241,4.39812) -- (7.66455,4.48852) -- (7.75669,4.57892) -- (7.84883,4.66931) -- (7.94098,4.75971) -- (8.03312,4.85011) -- (8.12526,4.9405) -- (8.21741,5.0309) -- (8.30955,5.1213) -- (8.40169,5.21169) --
 (8.49383,5.30209) -- (8.58598,5.39249) -- (8.67812,5.48288) -- (8.77026,5.57328) -- (8.8624,5.66368) -- (8.95455,5.75407) -- (9.04669,5.84447) -- (9.13883,5.93487) -- (9.23098,6.02526) -- (9.32312,6.11566) -- (9.41526,6.20606) -- (9.5074,6.29645) --
 (9.59955,6.38685) -- (9.69169,6.47725) -- (9.78383,6.56764) -- (9.87597,6.65804);
\draw [c,line width=0.9] (9.87597,6.65804) -- (9.96812,6.74844) -- (10.0603,6.83883) -- (10.1524,6.92923) -- (10.2445,7.01963) -- (10.3367,7.11002) -- (10.4288,7.20042) -- (10.521,7.29082) -- (10.6131,7.38121) -- (10.7053,7.47161) --
 (10.7974,7.56201) -- (10.8895,7.6524) -- (10.9817,7.7428) -- (11.0738,7.8332) -- (11.166,7.92359) -- (11.2581,8.01399) -- (11.3503,8.10439) -- (11.4424,8.19478) -- (11.5345,8.28518) -- (11.6267,8.37558) -- (11.7188,8.46597) -- (11.811,8.55637) --
 (11.9031,8.64677) -- (11.9953,8.73716) -- (12.0874,8.82756) -- (12.1795,8.91796) -- (12.2717,9.00835) -- (12.3638,9.09875) -- (12.456,9.18915) -- (12.5481,9.27954) -- (12.6403,9.36994) -- (12.7324,9.46034) -- (12.8245,9.55073) -- (12.9167,9.64113)
 -- (13.0088,9.73153) -- (13.101,9.82192) -- (13.1931,9.91232) -- (13.2853,10.0027) -- (13.3774,10.0931) -- (13.4695,10.1835) -- (13.5617,10.2739) -- (13.6538,10.3643) -- (13.746,10.4547) -- (13.8381,10.5451) -- (13.9303,10.6355) -- (14.0224,10.7259)
 -- (14.1145,10.8163) -- (14.2067,10.9067) -- (14.2988,10.9971) -- (14.391,11.0875);
\draw [c,line width=0.9] (14.391,11.0875) -- (14.4831,11.1779);
\definecolor{c}{rgb}{1,0,0};
\draw [c,line width=0.9] (13.2392,9.9241) -- (13.1528,9.9241);
\draw [c,line width=0.9] (13.1528,9.86663) -- (13.1528,9.98157);
\draw [c,line width=0.9] (13.2392,9.9241) -- (13.3255,9.9241);
\draw [c,line width=0.9] (13.3255,9.86663) -- (13.3255,9.98157);
\draw [c,line width=0.9] (13.2392,9.9241) -- (13.2392,10.0082);
\draw [c,line width=0.9] (13.1817,10.0082) -- (13.2967,10.0082);
\draw [c,line width=0.9] (13.2392,9.9241) -- (13.2392,9.83998);
\draw [c,line width=0.9] (13.1817,9.83998) -- (13.2967,9.83998);
\draw [c,line width=0.9] (6.72776,3.48638) -- (6.64328,3.48638);
\draw [c,line width=0.9] (6.64328,3.42891) -- (6.64328,3.54385);
\draw [c,line width=0.9] (6.72776,3.48638) -- (6.81225,3.48638);
\draw [c,line width=0.9] (6.81225,3.42891) -- (6.81225,3.54385);
\draw [c,line width=0.9] (6.72776,3.48638) -- (6.72776,3.5705);
\draw [c,line width=0.9] (6.67029,3.5705) -- (6.78524,3.5705);
\draw [c,line width=0.9] (6.72776,3.48638) -- (6.72776,3.40226);
\draw [c,line width=0.9] (6.67029,3.40226) -- (6.78524,3.40226);
\draw [c,line width=0.9] (10.2538,7.02712) -- (10.2451,7.02712);
\draw [c,line width=0.9] (10.2451,6.96965) -- (10.2451,7.0846);
\draw [c,line width=0.9] (10.2538,7.02712) -- (10.2624,7.02712);
\draw [c,line width=0.9] (10.2624,6.96965) -- (10.2624,7.0846);
\draw [c,line width=0.9] (10.2538,7.02712) -- (10.2538,7.03554);
\draw [c,line width=0.9] (10.1963,7.03554) -- (10.3112,7.03554);
\draw [c,line width=0.9] (10.2538,7.02712) -- (10.2538,7.01871);
\draw [c,line width=0.9] (10.1963,7.01871) -- (10.3112,7.01871);
\draw [c,line width=0.9] (9.5934,6.38335) -- (9.58481,6.38335);
\draw [c,line width=0.9] (9.58481,6.32588) -- (9.58481,6.44082);
\draw [c,line width=0.9] (9.5934,6.38335) -- (9.60199,6.38335);
\draw [c,line width=0.9] (9.60199,6.32588) -- (9.60199,6.44082);
\draw [c,line width=0.9] (9.5934,6.38335) -- (9.5934,6.39176);
\draw [c,line width=0.9] (9.53593,6.39176) -- (9.65087,6.39176);
\draw [c,line width=0.9] (9.5934,6.38335) -- (9.5934,6.37494);
\draw [c,line width=0.9] (9.53593,6.37494) -- (9.65087,6.37494);
\draw [c,line width=0.9] (11.2489,8.00482) -- (11.2144,8.00482);
\draw [c,line width=0.9] (11.2144,7.94734) -- (11.2144,8.06229);
\draw [c,line width=0.9] (11.2489,8.00482) -- (11.2834,8.00482);
\draw [c,line width=0.9] (11.2834,7.94734) -- (11.2834,8.06229);
\draw [c,line width=0.9] (11.2489,8.00482) -- (11.2489,8.03865);
\draw [c,line width=0.9] (11.1914,8.03865) -- (11.3064,8.03865);
\draw [c,line width=0.9] (11.2489,8.00482) -- (11.2489,7.97098);
\draw [c,line width=0.9] (11.1914,7.97098) -- (11.3064,7.97098);
\draw [c,line width=0.9] (8.61055,5.4207) -- (8.57624,5.4207);
\draw [c,line width=0.9] (8.57624,5.36323) -- (8.57624,5.47817);
\draw [c,line width=0.9] (8.61055,5.4207) -- (8.64486,5.4207);
\draw [c,line width=0.9] (8.64486,5.36323) -- (8.64486,5.47817);
\draw [c,line width=0.9] (8.61055,5.4207) -- (8.61055,5.45431);
\draw [c,line width=0.9] (8.55308,5.45431) -- (8.66802,5.45431);
\draw [c,line width=0.9] (8.61055,5.4207) -- (8.61055,5.3871);
\draw [c,line width=0.9] (8.55308,5.3871) -- (8.66802,5.3871);
\draw [c,line width=0.9] (12.2072,8.9434) -- (12.1485,8.9434);
\draw [c,line width=0.9] (12.1485,8.88593) -- (12.1485,9.00087);
\draw [c,line width=0.9] (12.2072,8.9434) -- (12.2659,8.9434);
\draw [c,line width=0.9] (12.2659,8.88593) -- (12.2659,9.00087);
\draw [c,line width=0.9] (12.2072,8.9434) -- (12.2072,9.00091);
\draw [c,line width=0.9] (12.1497,9.00091) -- (12.2647,9.00091);
\draw [c,line width=0.9] (12.2072,8.9434) -- (12.2072,8.88589);
\draw [c,line width=0.9] (12.1497,8.88589) -- (12.2647,8.88589);
\draw [c,line width=0.9] (7.63691,4.46708) -- (7.57819,4.46708);
\draw [c,line width=0.9] (7.57819,4.40961) -- (7.57819,4.52455);
\draw [c,line width=0.9] (7.63691,4.46708) -- (7.69562,4.46708);
\draw [c,line width=0.9] (7.69562,4.40961) -- (7.69562,4.52455);
\draw [c,line width=0.9] (7.63691,4.46708) -- (7.63691,4.52458);
\draw [c,line width=0.9] (7.57943,4.52458) -- (7.69438,4.52458);
\draw [c,line width=0.9] (7.63691,4.46708) -- (7.63691,4.40957);
\draw [c,line width=0.9] (7.57943,4.40957) -- (7.69438,4.40957);
\draw [c,line width=0.9] (14.1913,10.8867) -- (14.0785,10.8867);
\draw [c,line width=0.9] (14.0785,10.8293) -- (14.0785,10.9442);
\draw [c,line width=0.9] (14.1913,10.8867) -- (14.3041,10.8867);
\draw [c,line width=0.9] (14.3041,10.8293) -- (14.3041,10.9442);
\draw [c,line width=0.9] (14.1913,10.8867) -- (14.1913,10.9972);
\draw [c,line width=0.9] (14.1339,10.9972) -- (14.2488,10.9972);
\draw [c,line width=0.9] (14.1913,10.8867) -- (14.1913,10.7763);
\draw [c,line width=0.9] (14.1339,10.7763) -- (14.2488,10.7763);
\draw [c,line width=0.9] (5.68348,2.52373) -- (5.57114,2.52373);
\draw [c,line width=0.9] (5.57114,2.46626) -- (5.57114,2.5812);
\draw [c,line width=0.9] (5.68348,2.52373) -- (5.79582,2.52373);
\draw [c,line width=0.9] (5.79582,2.46626) -- (5.79582,2.5812);
\draw [c,line width=0.9] (5.68348,2.52373) -- (5.68348,2.63421);
\draw [c,line width=0.9] (5.62601,2.63421) -- (5.74095,2.63421);
\draw [c,line width=0.9] (5.68348,2.52373) -- (5.68348,2.41324);
\draw [c,line width=0.9] (5.62601,2.41324) -- (5.74095,2.41324);
\draw [c,line width=0.9] (15.2049,11.1876) -- (15.0647,11.1876);
\draw [c,line width=0.9] (15.0647,11.1301) -- (15.0647,11.245);
\draw [c,line width=0.9] (15.2049,11.1876) -- (15.3451,11.1876);
\draw [c,line width=0.9] (15.3451,11.1301) -- (15.3451,11.245);
\draw [c,line width=0.9] (15.2049,11.1876) -- (15.2049,11.3027);
\draw [c,line width=0.9] (15.1474,11.3027) -- (15.2624,11.3027);
\draw [c,line width=0.9] (15.2049,11.1876) -- (15.2049,11.0725);
\draw [c,line width=0.9] (15.1474,11.0725) -- (15.2624,11.0725);
\draw [c,line width=0.9] (4.70063,2.37331) -- (4.56136,2.37331);
\draw [c,line width=0.9] (4.56136,2.31584) -- (4.56136,2.43079);
\draw [c,line width=0.9] (4.70063,2.37331) -- (4.83989,2.37331);
\draw [c,line width=0.9] (4.83989,2.31584) -- (4.83989,2.43079);
\draw [c,line width=0.9] (4.70063,2.37331) -- (4.70063,2.48609);
\draw [c,line width=0.9] (4.64315,2.48609) -- (4.7581,2.48609);
\draw [c,line width=0.9] (4.70063,2.37331) -- (4.70063,2.26054);
\draw [c,line width=0.9] (4.64315,2.26054) -- (4.7581,2.26054);
\draw [c,line width=0.9] (16.1877,11.1274) -- (16.0206,11.1274);
\draw [c,line width=0.9] (16.0206,11.0699) -- (16.0206,11.1849);
\draw [c,line width=0.9] (16.1877,11.1274) -- (16.3549,11.1274);
\draw [c,line width=0.9] (16.3549,11.0699) -- (16.3549,11.1849);
\draw [c,line width=0.9] (16.1877,11.1274) -- (16.1877,11.2416);
\draw [c,line width=0.9] (16.1303,11.2416) -- (16.2452,11.2416);
\draw [c,line width=0.9] (16.1877,11.1274) -- (16.1877,11.0132);
\draw [c,line width=0.9] (16.1303,11.0132) -- (16.2452,11.0132);
\draw [c,line width=0.9] (3.80991,2.43348) -- (3.64508,2.43348);
\draw [c,line width=0.9] (3.64508,2.37601) -- (3.64508,2.49095);
\draw [c,line width=0.9] (3.80991,2.43348) -- (3.97474,2.43348);
\draw [c,line width=0.9] (3.97474,2.37601) -- (3.97474,2.49095);
\draw [c,line width=0.9] (3.80991,2.43348) -- (3.80991,2.54534);
\draw [c,line width=0.9] (3.75244,2.54534) -- (3.86738,2.54534);
\draw [c,line width=0.9] (3.80991,2.43348) -- (3.80991,2.32162);
\draw [c,line width=0.9] (3.75244,2.32162) -- (3.86738,2.32162);
\draw [c,line width=0.9] (16.4949,11.1876) -- (16.3231,11.1876);
\draw [c,line width=0.9] (16.3231,11.1301) -- (16.3231,11.245);
\draw [c,line width=0.9] (16.4949,11.1876) -- (16.6667,11.1876);
\draw [c,line width=0.9] (16.6667,11.1301) -- (16.6667,11.245);
\draw [c,line width=0.9] (16.4949,11.1876) -- (16.4949,11.3027);
\draw [c,line width=0.9] (16.4374,11.3027) -- (16.5524,11.3027);
\draw [c,line width=0.9] (16.4949,11.1876) -- (16.4949,11.0725);
\draw [c,line width=0.9] (16.4374,11.0725) -- (16.5524,11.0725);
\draw [c,line width=0.9] (3.50277,2.43348) -- (3.33333,2.43348);
\draw [c,line width=0.9] (3.33333,2.37601) -- (3.33333,2.49095);
\draw [c,line width=0.9] (3.50277,2.43348) -- (3.67221,2.43348);
\draw [c,line width=0.9] (3.67221,2.37601) -- (3.67221,2.49095);
\draw [c,line width=0.9] (3.50277,2.43348) -- (3.50277,2.54534);
\draw [c,line width=0.9] (3.4453,2.54534) -- (3.56024,2.54534);
\draw [c,line width=0.9] (3.50277,2.43348) -- (3.50277,2.32162);
\draw [c,line width=0.9] (3.4453,2.32162) -- (3.56024,2.32162);
\definecolor{c}{rgb}{1,1,1};
\draw [color=c, fill=c] (16,10.8506) rectangle (18,12.2069);
\definecolor{c}{rgb}{0,0,0};
\draw [c,line width=0.9] (16,10.8506) -- (18,10.8506);
\draw [c,line width=0.9] (18,10.8506) -- (18,12.2069);
\draw [c,line width=0.9] (18,12.2069) -- (16,12.2069);
\draw [c,line width=0.9] (16,12.2069) -- (16,10.8506);
\draw [anchor=base west] (16.5,11.7152) node[scale=1.02114, color=c, rotate=0]{Data};
\definecolor{c}{rgb}{1,0,0};
\draw [c,line width=0.9] (16.075,11.8678) -- (16.425,11.8678);
\definecolor{c}{rgb}{0,0,1};
\foreach \P in {(16.25,11.8678)}{\draw[mark options={color=c,fill=c},mark size=1.681682pt,mark=*,mark size=1pt] plot coordinates {\P};}
\definecolor{c}{rgb}{0,0,0};
\draw [anchor=base west] (16.5,11.0371) node[scale=1.02114, color=c, rotate=0]{Fit};
\definecolor{c}{rgb}{0,0,1};
\draw [c,line width=0.9] (16.075,11.1897) -- (16.425,11.1897);
\definecolor{c}{rgb}{0,0,0};
\draw (10,13.1224) node[scale=1.40406, color=c, rotate=0]{Gain};
\end{tikzpicture}

 }%
 \caption{Curva di trasferimento di un amplificatore invertente} 
 \label{gr:amp_noninv.tex} 
\end{grafico}

In seguito sono presentati i dati del  \autoref{gr:amp_noninv.tex} acquisiti in laboratorio, con i rispettivi errori:

\begin{tabella}
 \centering
 \begin{center}
\begin{tabulary}{\textwidth}{CCCCCC}
\toprule
$V_{in+}$ $\pm$ $\sigma_{V_{in+}}$ (V) &$V_{in-}$ $\pm$ $\sigma_{V_{in-}}$(V) & FS (V) & $V_{out+}$ $\pm$ $\sigma_{V_{out+}}(V)$ (V) & $V_{out-}$ $\pm$ $\sigma_{V_{out-}}$(V) & FS (V) \\ \midrule 
1.08$\pm$ 0.03 & -1.04$\pm$ 0.03 & 0.3 & 10.7$\pm$ 0.3 & -10.7$\pm$ 0.3 & 3 \\ \midrule
0.108$\pm$ 0.003 & -0.107$\pm$ 0.003 & 0.03 & 1.07$\pm$ 0.003 & -1.07$\pm$ 0.03 & 0.3 \\ \midrule
0.43$\pm$ 0.01 & -0.43$\pm$ 0.01 & 0.120 & 4.3$\pm$ 0.1 & -4.3$\pm$ 0.1 & 1.2 \\ \midrule
0.74$\pm$ 0.02 & -0.74$\pm$ 0.02 & 0.2 & 7.4$\pm$ 0.2 & -7.4$\pm$ 0.2 & 2 \\ \midrule 
1.39$\pm$ 0.04 & -1.38$\pm$ 0.04 & 0.4 & 13.9$\pm$ 0.4 & -13.9$\pm$ 0.4 & 4 \\ \midrule
1.72$\pm$ 0.05 & -1.70$\pm$ 0.05 & 0.5 & 14.9$\pm$ 0.4 & -14.4$\pm$ 0.4 & 4 \\ \midrule
2.04$\pm$ 0.05 & -1.99$\pm$ 0.05 & 0.6 & 14.7$\pm$ 0.4 & -14.2$\pm$ 0.4 & 4 \\ \midrule
2.14$\pm$ 0.06 & -2.09$\pm$ 0.06 & 0.6 & 14.9$\pm$ 0.4 & -14.2$\pm$ 0.4 & 4 \\ \midrule


\bottomrule
\end{tabulary}
\end{center}

 \caption{Dati curva di trasferimento}
 \label{tab:tab_non_inv.tex}
\end{tabella}
E' stata fatta l'interpolazione lineare pesata dei punti nel  \autoref{gr:amp_noninv.tex} compresi tra 0 e 1.5 V.\\
$q = -0.007 \pm 0.03 \, V$\\
$m = 10.0 \pm 0.1 \,\frac{V}{V}$



\FloatBarrier

\section{Parte II}
L'analisi dei dati che segue è stata effettuata utilizzando la macro fornite dal laboratorio,
modificando il limite dei campioni e inserendo il valore della baseline stimato in precedenza.

\subsection{Misure a pressione 650mb}
\FloatBarrier

\begin{grafico}
 \centering
 \resizebox{\textwidth}{!}{%
 \includegraphics{../grafici/risultati/misura650.png}
 }%
 \caption{Grafico segnali a 650mb} 
 \label{gr:misura_650} 
\end{grafico}

\begin{grafico}
 \centering
 \resizebox{\textwidth}{!}{%
 \includegraphics{../grafici/risultati/misura650_integral.png}
 }%
 \caption{Grafico integrale} 
 \label{gr:misura650_integral} 
\end{grafico}

 \begin{grafico}
 \centering
 \resizebox{\textwidth}{!}{%
 \includegraphics{../grafici/risultati/misura650_integral_ev.png}
 }%
 \caption{Grafico integral:ev} 
 \label{gr:misura650_integral_ev} 
\end{grafico}

 \begin{grafico}
 \centering
 \resizebox{\textwidth}{!}{%
 \includegraphics{../grafici/risultati/misura650_integral_vmax.png}
 }%
 \caption{Grafico integral:vmax} 
 \label{gr:misura650_integral_vmax} 
\end{grafico}

\begin{grafico}
 \centering
 \resizebox{\textwidth}{!}{%
 \includegraphics{../grafici/risultati/misura650_vmax.png}
 }%
 \caption{Grafico vmax} 
 \label{gr:misura650_vmax} 
\end{grafico}

 \begin{grafico}
 \centering
 \resizebox{\textwidth}{!}{%
 \includegraphics{../grafici/risultati/misura650_vmax_ev.png}
 }%
 \caption{Grafico vmax:ev} 
 \label{gr:misura650_vmax_ev} 
\end{grafico}

\FloatBarrier

\subsection{Misure a pressione 550mb}
\FloatBarrier

\begin{grafico}
 \centering
 \resizebox{\textwidth}{!}{%
 \includegraphics{../grafici/risultati/misura550.png}
 }%
 \caption{Grafico segnali a 550mb} 
 \label{gr:misura_550} 
\end{grafico}

\begin{grafico}
 \centering
 \resizebox{\textwidth}{!}{%
 \includegraphics{../grafici/risultati/misura550_integral.png}
 }%
 \caption{Grafico integrale} 
 \label{gr:misura550_integral} 
\end{grafico}

 \begin{grafico}
 \centering
 \resizebox{\textwidth}{!}{%
 \includegraphics{../grafici/risultati/misura550_integral_ev.png}
 }%
 \caption{Grafico integral:ev} 
 \label{gr:misura550_integral_ev} 
\end{grafico}

 \begin{grafico}
 \centering
 \resizebox{\textwidth}{!}{%
 \includegraphics{../grafici/risultati/misura550_integral_vmax.png}
 }%
 \caption{Grafico integral:vmax} 
 \label{gr:misura550_integral_vmax} 
\end{grafico}

\begin{grafico}
 \centering
 \resizebox{\textwidth}{!}{%
 \includegraphics{../grafici/risultati/misura550_vmax.png}
 }%
 \caption{Grafico vmax} 
 \label{gr:misura550_vmax} 
\end{grafico}

 \begin{grafico}
 \centering
 \resizebox{\textwidth}{!}{%
 \includegraphics{../grafici/risultati/misura550_vmax_ev.png}
 }%
 \caption{Grafico vmax:ev} 
 \label{gr:misura550_vmax_ev} 
\end{grafico}

\FloatBarrier

\subsection{Misure a pressione 500mb}
\FloatBarrier

\begin{grafico}[H]
 \centering
 \resizebox{\textwidth}{!}{%
 \includegraphics{../grafici/risultati/misura500.png}
 }%
 \caption{Grafico segnali a 500mb} 
 \label{gr:misura_500} 
\end{grafico}

\begin{grafico}
 \centering
 \resizebox{\textwidth}{!}{%
 \includegraphics{../grafici/risultati/misura500_integral.png}
 }%
 \caption{Grafico integrale} 
 \label{gr:misura500_integral} 
\end{grafico}

 \begin{grafico}
 \centering
 \resizebox{\textwidth}{!}{%
 \includegraphics{../grafici/risultati/misura500_integral_ev.png}
 }%
 \caption{Grafico integral:ev} 
 \label{gr:misura500_integral_ev} 
\end{grafico}

 \begin{grafico}
 \centering
 \resizebox{\textwidth}{!}{%
 \includegraphics{../grafici/risultati/misura500_integral_vmax.png}
 }%
 \caption{Grafico integral:vmax} 
 \label{gr:misura500_integral_vmax} 
\end{grafico}

\begin{grafico}
 \centering
 \resizebox{\textwidth}{!}{%
 \includegraphics{../grafici/risultati/misura500_vmax.png}
 }%
 \caption{Grafico vmax} 
 \label{gr:misura500_vmax} 
\end{grafico}

 \begin{grafico}
 \centering
 \resizebox{\textwidth}{!}{%
 \includegraphics{../grafici/risultati/misura500_vmax_ev.png}
 }%
 \caption{Grafico vmax:ev} 
 \label{gr:misura500_vmax_ev} 
\end{grafico}

\FloatBarrier

\subsection{Misure a pressione 450mb}
\FloatBarrier

\begin{grafico}
 \centering
 \resizebox{\textwidth}{!}{%
 \includegraphics{../grafici/risultati/misura450.png}
 }%
 \caption{Grafico segnali a 450mb} 
 \label{gr:misura_450} 
\end{grafico}

\begin{grafico}
 \centering
 \resizebox{\textwidth}{!}{%
 \includegraphics{../grafici/risultati/misura450_integral.png}
 }%
 \caption{Grafico integrale} 
 \label{gr:misura450_integral} 
\end{grafico}

 \begin{grafico}
 \centering
 \resizebox{\textwidth}{!}{%
 \includegraphics{../grafici/risultati/misura450_integral_ev.png}
 }%
 \caption{Grafico integral:ev} 
 \label{gr:misura450_integral_ev} 
\end{grafico}

 \begin{grafico}
 \centering
 \resizebox{\textwidth}{!}{%
 \includegraphics{../grafici/risultati/misura450_integral_vmax.png}
 }%
 \caption{Grafico integral:vmax} 
 \label{gr:misura450_integral_vmax} 
\end{grafico}

\begin{grafico}
 \centering
 \resizebox{\textwidth}{!}{%
 \includegraphics{../grafici/risultati/misura450_vmax.png}
 }%
 \caption{Grafico vmax} 
 \label{gr:misura450_vmax} 
\end{grafico}

 \begin{grafico}
 \centering
 \resizebox{\textwidth}{!}{%
 \includegraphics{../grafici/risultati/misura450_vmax_ev.png}
 }%
 \caption{Grafico vmax:ev} 
 \label{gr:misura450_vmax_ev} 
\end{grafico}

\FloatBarrier

\subsection{Misure a pressione 400mb}
\FloatBarrier

\begin{grafico}[H]
 \centering
 \resizebox{\textwidth}{!}{%
 \includegraphics{../grafici/risultati/misura400.png}
 }%
 \caption{Grafico segnali a 400mb} 
 \label{gr:misura_400} 
\end{grafico}
\begin{grafico}
 \centering
 \resizebox{\textwidth}{!}{%
 \includegraphics{../grafici/risultati/misura400_integral.png}
 }%
 \caption{Grafico integrale} 
 \label{gr:misura400_integral} 
\end{grafico}

 \begin{grafico}
 \centering
 \resizebox{\textwidth}{!}{%
 \includegraphics{../grafici/risultati/misura400_integral_ev.png}
 }%
 \caption{Grafico integral:ev} 
 \label{gr:misura400_integral_ev} 
\end{grafico}

 \begin{grafico}
 \centering
 \resizebox{\textwidth}{!}{%
 \includegraphics{../grafici/risultati/misura400_integral_vmax.png}
 }%
 \caption{Grafico integral:vmax} 
 \label{gr:misura400_integral_vmax} 
\end{grafico}

\begin{grafico}
 \centering
 \resizebox{\textwidth}{!}{%
 \includegraphics{../grafici/risultati/misura400_vmax.png}
 }%
 \caption{Grafico vmax} 
 \label{gr:misura400_vmax} 
\end{grafico}

 \begin{grafico}
 \centering
 \resizebox{\textwidth}{!}{%
 \includegraphics{../grafici/risultati/misura400_vmax_ev.png}
 }%
 \caption{Grafico vmax:ev} 
 \label{gr:misura400_vmax_ev} 
\end{grafico}

\FloatBarrier

\subsection{Misure a pressione 380mb}
\FloatBarrier

\begin{grafico}[H]
 \centering
 \resizebox{\textwidth}{!}{%
 \includegraphics{../grafici/risultati/misura380.png}
 }%
 \caption{Grafico segnali a 380mb} 
 \label{gr:misura_380} 
\end{grafico}

\begin{grafico}[H]
 \centering
 \resizebox{\textwidth}{!}{%
 \includegraphics{../grafici/risultati/misura380_integral.png}
 }%
 \caption{Grafico integrale} 
 \label{gr:misura380_integral} 
\end{grafico}

 \begin{grafico}
 \centering
 \resizebox{\textwidth}{!}{%
 \includegraphics{../grafici/risultati/misura380_integral_ev.png}
 }%
 \caption{Grafico integral:ev} 
 \label{gr:misura380_integral_ev} 
\end{grafico}

 \begin{grafico}
 \centering
 \resizebox{\textwidth}{!}{%
 \includegraphics{../grafici/risultati/misura380_integral_vmax.png}
 }%
 \caption{Grafico integral:vmax} 
 \label{gr:misura380_integral_vmax} 
\end{grafico}

\begin{grafico}
 \centering
 \resizebox{\textwidth}{!}{%
 \includegraphics{../grafici/risultati/misura380_vmax.png}
 }%
 \caption{Grafico vmax} 
 \label{gr:misura380_vmax} 
\end{grafico}

 \begin{grafico}
 \centering
 \resizebox{\textwidth}{!}{%
 \includegraphics{../grafici/risultati/misura380_vmax_ev.png}
 }%
 \caption{Grafico vmax:ev} 
 \label{gr:misura380_vmax_ev} 
\end{grafico}


\FloatBarrier

\subsection{Curve in funzione della pressione}
\FloatBarrier
Sono state registrate le posizioni dei centroidi e l'RMS dei picchi in energia e dei picchi di vmax delle alfa delle sorgenti.

\begin{tabella}
 \centering
 \input{tabelle/tab_integral1.tex} 
 \caption{Dati picchi in energia e RMS}
 \label{tab:tab_integral1.tex}
\end{tabella}

\begin{tabella}
 \centering
 \input{tabelle/tab_integral2.tex} 
 \caption{Dati picchi in energia e RMS}
 \label{tab:tab_integral2.tex}
\end{tabella}

\begin{tabella}
 \centering
 \input{tabelle/tab_integral3.tex} 
 \caption{Dati picchi in energia e RMS}
 \label{tab:tab_integral3.tex}
\end{tabella}

\begin{tabella}
 \centering
 \input{tabelle/tab_vmax.tex} 
 \caption{Dati vmax in emergia e RMS}
 \label{tab:tab_vmax.tex}
\end{tabella}




In seguito sono stati riportati in grafico i centroidi delle energie e i centroidi dei massimi in funzione della pressione:

\begin{grafico}
 \centering
 \resizebox{\textwidth}{!}{%
 \includegraphics{../grafici/risultati/picchi_integral1.pdf}
 }%
 \caption{Andamento integrale del picco 1 in funzione della pressione [mb]} 
 \label{gr:picchi_int1} 
\end{grafico}

\begin{grafico}
 \centering
 \resizebox{\textwidth}{!}{%
 \includegraphics{../grafici/risultati/picchi_integral2.pdf}
 }%
 \caption{Andamento integrale del picco 2 in funzione della pressione [mb]} 
 \label{gr:picchi_int2} 
\end{grafico}

\begin{grafico}
 \centering
 \resizebox{\textwidth}{!}{%
 \includegraphics{../grafici/risultati/picchi_integral3.pdf}
 }%
 \caption{Andamento integrale del picco 3 in funzione della pressione [mb]} 
 \label{gr:picchi_int3} 
\end{grafico}

\begin{grafico}
 \centering
 \resizebox{\textwidth}{!}{%
 \includegraphics{../grafici/risultati/picchi_vmax.png}
 }%
 \caption{Andamento integrale di vmax [V] in funzione della pressione [mb]} 
 \label{gr:picchi_vmax} 
\end{grafico}

Come si può vedere dai grafici 	\autoref{gr:picchi_int1}, \autoref{gr:picchi_int2} e \autoref{gr:picchi_int3}, l'integrale dei picchi, ovvero l'energia della particella alfa, rimane costante al variare della pressione,
almeno finche non si arriva a pressioni troppo basse. Questo cambiamento a pressioni basse è dovuto al fatto che a bassa pressione le particelle riescono ad arrivare oltre la lunghezza della camera, e la restante carica non viene più rivelata.
Addirittura, il terzo picco sparisce a basse pressioni, confondendosi con il secondo a causa dell'energia mancante.

Nel grafico \autoref{gr:picchi_vmax} invece si può chiaramente notare un andamento lineare (TODO perché?).

\FloatBarrier
\FloatBarrier

	%\subsection{Tabelle}

	%\FloatBarrier
%\subsection{Tabelle dati}

\begin{tabella}
 \centering
 \begin{center}
\begin{tabulary}{\textwidth}{CC}
\toprule
 f(Hz) & A \\ \midrule
 10 & $1.01\pm0.03$   \\ \midrule
 30 & $1.01\pm0.03$  \\ \midrule
 100 & $1.01\pm0.03$  \\ \midrule
 300 & $1.01\pm0.03$  \\ \midrule
 1000 & $1.01\pm0.03$ \\ \midrule
 3000 & $1.00\pm0.03$   \\ \midrule
 10000 & $1.00\pm0.03$  \\ \midrule
 30000 & $1.01\pm0.03$  \\ \midrule
 100000 & $1.02\pm0.03$  \\ \midrule
 300000 & $1.08\pm0.04$  \\ \midrule
 800000 & $1.18\pm0.04$  \\ \midrule
 900000 & $1.05\pm0.03$  \\ \midrule
 1000000 & $0.90\pm0.03$ \\ \midrule
1130000 & $0.71\pm0.02$  \\ \midrule
1500000 & $0.39\pm0.01$  \\ \midrule
2000000 & $0.210\pm0.007$  \\ \midrule
3000000 & $0.090\pm0.003$  \\ \midrule
6000000 & $0.030\pm0.001$  \\ \midrule
 \bottomrule
\end{tabulary}
\end{center}


 
 \caption{Dati risposta in frequenza}
 \label{tab:tab_noninv_A1.tex}
\end{tabella}

\begin{tabella}
 \centering
  \begin{center}
\begin{tabulary}{\textwidth}{CC}
\toprule
f(Hz) & A  \\ \midrule
10 & 4.9$\pm$0.2  \\ \midrule
40 & 4.9$\pm$0.2  \\ \midrule
100 & 4.9$\pm$0.2  \\ \midrule
300 & 4.9$\pm$0.2  \\ \midrule
1000 & 4.9$\pm$0.2  \\ \midrule
3000 & 4.9$\pm$0.2  \\ \midrule
10000 & 4.9$\pm$0.2 \\ \midrule
30000 & 4.9$\pm$0.2  \\ \midrule
100000 & 4.8$\pm$0.2  \\ \midrule
300000 & 4.3$\pm$0.1  \\ \midrule
400000 & 4.0$\pm$0.1  \\ \midrule
515000 & 3.5$\pm$0.1   \\ \midrule
600000 & 3.2$\pm$0.1   \\ \midrule
1000000 & 2.02$\pm$0.07  \\ \midrule
3000000 & 0.44$\pm$0.01  \\ \midrule
5000000 & 0.170$\pm$0.006  \\ \midrule
 \bottomrule
\end{tabulary}
\end{center}



 
 \caption{Dati risposta in frequenza}
 \label{tab:tab_noninv_A5.tex}
\end{tabella}

\begin{tabella}
 \centering
 \begin{center}
\begin{tabulary}{\textwidth}{CC}
\toprule
f(Hz) & A   \\ \midrule
10 & 10.1$\pm$0.3  \\ \midrule
50 & 10.1$\pm$0.3  \\ \midrule
100 & 10.1$\pm$0.3  \\ \midrule
500 & 10.1$\pm$0.3  \\ \midrule
1000 & 10.1$\pm$0.3  \\ \midrule
5000 & 10.1$\pm$0.3  \\ \midrule
50000 & 9.4$\pm$0.3  \\ \midrule
100000 & 9.2$\pm$0.3  \\ \midrule
200000 & 7.5$\pm$0.3  \\ \midrule
211000 & 7.1$\pm$0.2  \\ \midrule
215000 & 7.1$\pm$0.2  \\ \midrule
220000 & 7.1$\pm$0.2  \\ \midrule
230000 & 6.9$\pm$0.2  \\ \midrule
250000 & 6.5$\pm$0.2  \\ \midrule
300000 & 5.9$\pm$0.2  \\ \midrule
400000 & 4.7$\pm$0.2  \\ \midrule
500000 & 3.69$\pm$0.1  \\ \midrule
1000000 & 1.57$\pm$0.05  \\ \midrule
5000000 & 0.110$\pm$0.003  \\ \midrule
\bottomrule
\end{tabulary}
\end{center} 
 \caption{Dati risposta in frequenza}
 \label{tab:tab_noninv_A10.tex}
\end{tabella}

\FloatBarrier



	%\clearpage
	%\subsection{Grafici}
	%% \begin{grafico} \centering \input{../gnuplot/immagini/02_graph_1.tex} \caption{Grafico 02_graph_1.tex} \label{gr:02_graph_1.tex} \end{grafico}
% \begin{grafico} \centering \input{../gnuplot/immagini/02_graph_3.tex} \caption{Grafico 02_graph_3.tex} \label{gr:02_graph_3.tex} \end{grafico}
% \begin{grafico} \centering \input{../gnuplot/immagini/02_graph_4.tex} \caption{Grafico 02_graph_4.tex} \label{gr:02_graph_4.tex} \end{grafico}


%\clearpage
%\begin{multicols}{2}

	
\section{Discussioni e conclusioni}
	E' stata fatta la simulazione per la risposta in frequenza utilizzando Spice, con l'accortezza di scaricare la libreria del componente
dal sito della Texas Instrument.
I  \autoref{gr:sim_a5.tex} e \autoref{gr:sim_a10.tex} mostrano il confronto della simulazione con 
i dati sperimentali e mostrano un buon accordo tra i dati e la simulazione.
Nel \autoref{gr:sim_a1.tex}, relativo all'ampificazione A=1, si nota una deviazione dei dati sperimentali dovuta alla risonanza,
causata dalle induttanze parassite nel circuito reale, che comportano lo spostamento verso le basse frequenze del polo.

\begin{grafico}
 \centering
 \resizebox{\textwidth}{!}{%
 \begin{tikzpicture}
\pgfdeclareplotmark{cross} {
\pgfpathmoveto{\pgfpoint{-0.3\pgfplotmarksize}{\pgfplotmarksize}}
\pgfpathlineto{\pgfpoint{+0.3\pgfplotmarksize}{\pgfplotmarksize}}
\pgfpathlineto{\pgfpoint{+0.3\pgfplotmarksize}{0.3\pgfplotmarksize}}
\pgfpathlineto{\pgfpoint{+1\pgfplotmarksize}{0.3\pgfplotmarksize}}
\pgfpathlineto{\pgfpoint{+1\pgfplotmarksize}{-0.3\pgfplotmarksize}}
\pgfpathlineto{\pgfpoint{+0.3\pgfplotmarksize}{-0.3\pgfplotmarksize}}
\pgfpathlineto{\pgfpoint{+0.3\pgfplotmarksize}{-1.\pgfplotmarksize}}
\pgfpathlineto{\pgfpoint{-0.3\pgfplotmarksize}{-1.\pgfplotmarksize}}
\pgfpathlineto{\pgfpoint{-0.3\pgfplotmarksize}{-0.3\pgfplotmarksize}}
\pgfpathlineto{\pgfpoint{-1.\pgfplotmarksize}{-0.3\pgfplotmarksize}}
\pgfpathlineto{\pgfpoint{-1.\pgfplotmarksize}{0.3\pgfplotmarksize}}
\pgfpathlineto{\pgfpoint{-0.3\pgfplotmarksize}{0.3\pgfplotmarksize}}
\pgfpathclose
\pgfusepathqstroke
}
\pgfdeclareplotmark{cross*} {
\pgfpathmoveto{\pgfpoint{-0.3\pgfplotmarksize}{\pgfplotmarksize}}
\pgfpathlineto{\pgfpoint{+0.3\pgfplotmarksize}{\pgfplotmarksize}}
\pgfpathlineto{\pgfpoint{+0.3\pgfplotmarksize}{0.3\pgfplotmarksize}}
\pgfpathlineto{\pgfpoint{+1\pgfplotmarksize}{0.3\pgfplotmarksize}}
\pgfpathlineto{\pgfpoint{+1\pgfplotmarksize}{-0.3\pgfplotmarksize}}
\pgfpathlineto{\pgfpoint{+0.3\pgfplotmarksize}{-0.3\pgfplotmarksize}}
\pgfpathlineto{\pgfpoint{+0.3\pgfplotmarksize}{-1.\pgfplotmarksize}}
\pgfpathlineto{\pgfpoint{-0.3\pgfplotmarksize}{-1.\pgfplotmarksize}}
\pgfpathlineto{\pgfpoint{-0.3\pgfplotmarksize}{-0.3\pgfplotmarksize}}
\pgfpathlineto{\pgfpoint{-1.\pgfplotmarksize}{-0.3\pgfplotmarksize}}
\pgfpathlineto{\pgfpoint{-1.\pgfplotmarksize}{0.3\pgfplotmarksize}}
\pgfpathlineto{\pgfpoint{-0.3\pgfplotmarksize}{0.3\pgfplotmarksize}}
\pgfpathclose
\pgfusepathqfillstroke
}
\pgfdeclareplotmark{newstar} {
\pgfpathmoveto{\pgfqpoint{0pt}{\pgfplotmarksize}}
\pgfpathlineto{\pgfqpointpolar{44}{0.5\pgfplotmarksize}}
\pgfpathlineto{\pgfqpointpolar{18}{\pgfplotmarksize}}
\pgfpathlineto{\pgfqpointpolar{-20}{0.5\pgfplotmarksize}}
\pgfpathlineto{\pgfqpointpolar{-54}{\pgfplotmarksize}}
\pgfpathlineto{\pgfqpointpolar{-90}{0.5\pgfplotmarksize}}
\pgfpathlineto{\pgfqpointpolar{234}{\pgfplotmarksize}}
\pgfpathlineto{\pgfqpointpolar{198}{0.5\pgfplotmarksize}}
\pgfpathlineto{\pgfqpointpolar{162}{\pgfplotmarksize}}
\pgfpathlineto{\pgfqpointpolar{134}{0.5\pgfplotmarksize}}
\pgfpathclose
\pgfusepathqstroke
}
\pgfdeclareplotmark{newstar*} {
\pgfpathmoveto{\pgfqpoint{0pt}{\pgfplotmarksize}}
\pgfpathlineto{\pgfqpointpolar{44}{0.5\pgfplotmarksize}}
\pgfpathlineto{\pgfqpointpolar{18}{\pgfplotmarksize}}
\pgfpathlineto{\pgfqpointpolar{-20}{0.5\pgfplotmarksize}}
\pgfpathlineto{\pgfqpointpolar{-54}{\pgfplotmarksize}}
\pgfpathlineto{\pgfqpointpolar{-90}{0.5\pgfplotmarksize}}
\pgfpathlineto{\pgfqpointpolar{234}{\pgfplotmarksize}}
\pgfpathlineto{\pgfqpointpolar{198}{0.5\pgfplotmarksize}}
\pgfpathlineto{\pgfqpointpolar{162}{\pgfplotmarksize}}
\pgfpathlineto{\pgfqpointpolar{134}{0.5\pgfplotmarksize}}
\pgfpathclose
\pgfusepathqfillstroke
}
\definecolor{c}{rgb}{1,1,1};
\draw [color=c, fill=c] (0,0) rectangle (20,13.4957);
\draw [color=c, fill=c] (2,1.34957) rectangle (18,12.1461);
\definecolor{c}{rgb}{0,0,0};
\draw [c,line width=0.9] (2,1.34957) -- (2,12.1461) -- (18,12.1461) -- (18,1.34957) -- (2,1.34957);
\definecolor{c}{rgb}{1,1,1};
\draw [color=c, fill=c] (2,1.34957) rectangle (18,12.1461);
\definecolor{c}{rgb}{0,0,0};
\draw [c,line width=0.9] (2,1.34957) -- (2,12.1461) -- (18,12.1461) -- (18,1.34957) -- (2,1.34957);
\draw [c,line width=0.9] (2,1.34957) -- (18,1.34957);
\draw [c,dotted,line width=0.9] (2.09053,12.1461) -- (2.09053,1.34957);
\draw [c,dotted,line width=0.9] (4.06897,12.1461) -- (4.06897,1.34957);
\draw [c,dotted,line width=0.9] (6.04742,12.1461) -- (6.04742,1.34957);
\draw [c,dotted,line width=0.9] (8.02587,12.1461) -- (8.02587,1.34957);
\draw [c,dotted,line width=0.9] (10.0043,12.1461) -- (10.0043,1.34957);
\draw [c,dotted,line width=0.9] (11.9828,12.1461) -- (11.9828,1.34957);
\draw [c,dotted,line width=0.9] (13.9612,12.1461) -- (13.9612,1.34957);
\draw [c,dotted,line width=0.9] (15.9397,12.1461) -- (15.9397,1.34957);
\draw [c,dotted,line width=0.9] (17.9181,12.1461) -- (17.9181,1.34957);
\draw [c,line width=0.9] (2,1.34957) -- (2,12.1461);
\draw [c,dotted,line width=0.9] (18,2.19381) -- (2,2.19381);
\draw [c,dotted,line width=0.9] (18,2.82754) -- (2,2.82754);
\draw [c,dotted,line width=0.9] (18,3.27717) -- (2,3.27717);
\draw [c,dotted,line width=0.9] (18,3.62594) -- (2,3.62594);
\draw [c,dotted,line width=0.9] (18,3.9109) -- (2,3.9109);
\draw [c,dotted,line width=0.9] (18,4.15183) -- (2,4.15183);
\draw [c,dotted,line width=0.9] (18,4.36054) -- (2,4.36054);
\draw [c,dotted,line width=0.9] (18,4.54463) -- (2,4.54463);
\draw [c,dotted,line width=0.9] (18,4.7093) -- (2,4.7093);
\draw [c,dotted,line width=0.9] (18,5.79266) -- (2,5.79266);
\draw [c,dotted,line width=0.9] (18,6.42639) -- (2,6.42639);
\draw [c,dotted,line width=0.9] (18,6.87603) -- (2,6.87603);
\draw [c,dotted,line width=0.9] (18,7.22479) -- (2,7.22479);
\draw [c,dotted,line width=0.9] (18,7.50975) -- (2,7.50975);
\draw [c,dotted,line width=0.9] (18,7.75068) -- (2,7.75068);
\draw [c,dotted,line width=0.9] (18,7.95939) -- (2,7.95939);
\draw [c,dotted,line width=0.9] (18,8.14348) -- (2,8.14348);
\draw [c,dotted,line width=0.9] (18,8.30815) -- (2,8.30815);
\draw [c,dotted,line width=0.9] (18,9.39151) -- (2,9.39151);
\draw [c,dotted,line width=0.9] (18,10.0252) -- (2,10.0252);
\draw [c,dotted,line width=0.9] (18,10.4749) -- (2,10.4749);
\draw [c,dotted,line width=0.9] (18,10.8236) -- (2,10.8236);
\draw [c,dotted,line width=0.9] (18,11.1086) -- (2,11.1086);
\draw [c,dotted,line width=0.9] (18,11.3495) -- (2,11.3495);
\draw [c,dotted,line width=0.9] (18,11.5582) -- (2,11.5582);
\draw [c,dotted,line width=0.9] (18,11.7423) -- (2,11.7423);
\draw [c,dotted,line width=0.9] (18,11.907) -- (2,11.907);
\draw [c,line width=0.9] (2,1.34957) -- (18,1.34957);
\draw [c,line width=0.9] (2.09053,1.67347) -- (2.09053,1.34957);
\draw [anchor=base] (2.09053,0.73889) node[scale=1.01821, color=c, rotate=0]{10};
\draw [c,line width=0.9] (2.6861,1.51152) -- (2.6861,1.34957);
\draw [c,line width=0.9] (3.03449,1.51152) -- (3.03449,1.34957);
\draw [c,line width=0.9] (3.28167,1.51152) -- (3.28167,1.34957);
\draw [c,line width=0.9] (3.4734,1.51152) -- (3.4734,1.34957);
\draw [c,line width=0.9] (3.63006,1.51152) -- (3.63006,1.34957);
\draw [c,line width=0.9] (3.76251,1.51152) -- (3.76251,1.34957);
\draw [c,line width=0.9] (3.87724,1.51152) -- (3.87724,1.34957);
\draw [c,line width=0.9] (3.97845,1.51152) -- (3.97845,1.34957);
\draw [c,line width=0.9] (4.06897,1.67347) -- (4.06897,1.34957);
\draw [anchor=base] (4.06897,0.73889) node[scale=1.01821, color=c, rotate=0]{$10^{2}$};
\draw [c,line width=0.9] (4.66455,1.51152) -- (4.66455,1.34957);
\draw [c,line width=0.9] (5.01293,1.51152) -- (5.01293,1.34957);
\draw [c,line width=0.9] (5.26012,1.51152) -- (5.26012,1.34957);
\draw [c,line width=0.9] (5.45185,1.51152) -- (5.45185,1.34957);
\draw [c,line width=0.9] (5.60851,1.51152) -- (5.60851,1.34957);
\draw [c,line width=0.9] (5.74096,1.51152) -- (5.74096,1.34957);
\draw [c,line width=0.9] (5.85569,1.51152) -- (5.85569,1.34957);
\draw [c,line width=0.9] (5.95689,1.51152) -- (5.95689,1.34957);
\draw [c,line width=0.9] (6.04742,1.67347) -- (6.04742,1.34957);
\draw [anchor=base] (6.04742,0.73889) node[scale=1.01821, color=c, rotate=0]{$10^{3}$};
\draw [c,line width=0.9] (6.64299,1.51152) -- (6.64299,1.34957);
\draw [c,line width=0.9] (6.99138,1.51152) -- (6.99138,1.34957);
\draw [c,line width=0.9] (7.23857,1.51152) -- (7.23857,1.34957);
\draw [c,line width=0.9] (7.4303,1.51152) -- (7.4303,1.34957);
\draw [c,line width=0.9] (7.58695,1.51152) -- (7.58695,1.34957);
\draw [c,line width=0.9] (7.7194,1.51152) -- (7.7194,1.34957);
\draw [c,line width=0.9] (7.83414,1.51152) -- (7.83414,1.34957);
\draw [c,line width=0.9] (7.93534,1.51152) -- (7.93534,1.34957);
\draw [c,line width=0.9] (8.02587,1.67347) -- (8.02587,1.34957);
\draw [anchor=base] (8.02587,0.73889) node[scale=1.01821, color=c, rotate=0]{$10^{4}$};
\draw [c,line width=0.9] (8.62144,1.51152) -- (8.62144,1.34957);
\draw [c,line width=0.9] (8.96983,1.51152) -- (8.96983,1.34957);
\draw [c,line width=0.9] (9.21701,1.51152) -- (9.21701,1.34957);
\draw [c,line width=0.9] (9.40874,1.51152) -- (9.40874,1.34957);
\draw [c,line width=0.9] (9.5654,1.51152) -- (9.5654,1.34957);
\draw [c,line width=0.9] (9.69785,1.51152) -- (9.69785,1.34957);
\draw [c,line width=0.9] (9.81259,1.51152) -- (9.81259,1.34957);
\draw [c,line width=0.9] (9.91379,1.51152) -- (9.91379,1.34957);
\draw [c,line width=0.9] (10.0043,1.67347) -- (10.0043,1.34957);
\draw [anchor=base] (10.0043,0.73889) node[scale=1.01821, color=c, rotate=0]{$10^{5}$};
\draw [c,line width=0.9] (10.5999,1.51152) -- (10.5999,1.34957);
\draw [c,line width=0.9] (10.9483,1.51152) -- (10.9483,1.34957);
\draw [c,line width=0.9] (11.1955,1.51152) -- (11.1955,1.34957);
\draw [c,line width=0.9] (11.3872,1.51152) -- (11.3872,1.34957);
\draw [c,line width=0.9] (11.5438,1.51152) -- (11.5438,1.34957);
\draw [c,line width=0.9] (11.6763,1.51152) -- (11.6763,1.34957);
\draw [c,line width=0.9] (11.791,1.51152) -- (11.791,1.34957);
\draw [c,line width=0.9] (11.8922,1.51152) -- (11.8922,1.34957);
\draw [c,line width=0.9] (11.9828,1.67347) -- (11.9828,1.34957);
\draw [anchor=base] (11.9828,0.73889) node[scale=1.01821, color=c, rotate=0]{$10^{6}$};
\draw [c,line width=0.9] (12.5783,1.51152) -- (12.5783,1.34957);
\draw [c,line width=0.9] (12.9267,1.51152) -- (12.9267,1.34957);
\draw [c,line width=0.9] (13.1739,1.51152) -- (13.1739,1.34957);
\draw [c,line width=0.9] (13.3656,1.51152) -- (13.3656,1.34957);
\draw [c,line width=0.9] (13.5223,1.51152) -- (13.5223,1.34957);
\draw [c,line width=0.9] (13.6547,1.51152) -- (13.6547,1.34957);
\draw [c,line width=0.9] (13.7695,1.51152) -- (13.7695,1.34957);
\draw [c,line width=0.9] (13.8707,1.51152) -- (13.8707,1.34957);
\draw [c,line width=0.9] (13.9612,1.67347) -- (13.9612,1.34957);
\draw [anchor=base] (13.9612,0.73889) node[scale=1.01821, color=c, rotate=0]{$10^{7}$};
\draw [c,line width=0.9] (14.5568,1.51152) -- (14.5568,1.34957);
\draw [c,line width=0.9] (14.9052,1.51152) -- (14.9052,1.34957);
\draw [c,line width=0.9] (15.1524,1.51152) -- (15.1524,1.34957);
\draw [c,line width=0.9] (15.3441,1.51152) -- (15.3441,1.34957);
\draw [c,line width=0.9] (15.5007,1.51152) -- (15.5007,1.34957);
\draw [c,line width=0.9] (15.6332,1.51152) -- (15.6332,1.34957);
\draw [c,line width=0.9] (15.7479,1.51152) -- (15.7479,1.34957);
\draw [c,line width=0.9] (15.8491,1.51152) -- (15.8491,1.34957);
\draw [c,line width=0.9] (15.9397,1.67347) -- (15.9397,1.34957);
\draw [anchor=base] (15.9397,0.73889) node[scale=1.01821, color=c, rotate=0]{$10^{8}$};
\draw [c,line width=0.9] (16.5352,1.51152) -- (16.5352,1.34957);
\draw [c,line width=0.9] (16.8836,1.51152) -- (16.8836,1.34957);
\draw [c,line width=0.9] (17.1308,1.51152) -- (17.1308,1.34957);
\draw [c,line width=0.9] (17.3225,1.51152) -- (17.3225,1.34957);
\draw [c,line width=0.9] (17.4792,1.51152) -- (17.4792,1.34957);
\draw [c,line width=0.9] (17.6116,1.51152) -- (17.6116,1.34957);
\draw [c,line width=0.9] (17.7264,1.51152) -- (17.7264,1.34957);
\draw [c,line width=0.9] (17.8276,1.51152) -- (17.8276,1.34957);
\draw [c,line width=0.9] (17.9181,1.67347) -- (17.9181,1.34957);
\draw [anchor=base] (17.9181,0.73889) node[scale=1.01821, color=c, rotate=0]{$10^{9}$};
\draw [c,line width=0.9] (2,1.34957) -- (2,12.1461);
\draw [c,line width=0.9] (2.24,2.19381) -- (2,2.19381);
\draw [c,line width=0.9] (2.24,2.82754) -- (2,2.82754);
\draw [c,line width=0.9] (2.24,3.27717) -- (2,3.27717);
\draw [c,line width=0.9] (2.24,3.62594) -- (2,3.62594);
\draw [c,line width=0.9] (2.24,3.9109) -- (2,3.9109);
\draw [c,line width=0.9] (2.24,4.15183) -- (2,4.15183);
\draw [c,line width=0.9] (2.24,4.36054) -- (2,4.36054);
\draw [c,line width=0.9] (2.24,4.54463) -- (2,4.54463);
\draw [c,line width=0.9] (2.48,4.7093) -- (2,4.7093);
\draw [anchor= east] (1.844,4.7093) node[scale=1.01821, color=c, rotate=0]{$10^{-2}$};
\draw [c,line width=0.9] (2.24,5.79266) -- (2,5.79266);
\draw [c,line width=0.9] (2.24,6.42639) -- (2,6.42639);
\draw [c,line width=0.9] (2.24,6.87603) -- (2,6.87603);
\draw [c,line width=0.9] (2.24,7.22479) -- (2,7.22479);
\draw [c,line width=0.9] (2.24,7.50975) -- (2,7.50975);
\draw [c,line width=0.9] (2.24,7.75068) -- (2,7.75068);
\draw [c,line width=0.9] (2.24,7.95939) -- (2,7.95939);
\draw [c,line width=0.9] (2.24,8.14348) -- (2,8.14348);
\draw [c,line width=0.9] (2.48,8.30815) -- (2,8.30815);
\draw [anchor= east] (1.844,8.30815) node[scale=1.01821, color=c, rotate=0]{$10^{-1}$};
\draw [c,line width=0.9] (2.24,9.39151) -- (2,9.39151);
\draw [c,line width=0.9] (2.24,10.0252) -- (2,10.0252);
\draw [c,line width=0.9] (2.24,10.4749) -- (2,10.4749);
\draw [c,line width=0.9] (2.24,10.8236) -- (2,10.8236);
\draw [c,line width=0.9] (2.24,11.1086) -- (2,11.1086);
\draw [c,line width=0.9] (2.24,11.3495) -- (2,11.3495);
\draw [c,line width=0.9] (2.24,11.5582) -- (2,11.5582);
\draw [c,line width=0.9] (2.24,11.7423) -- (2,11.7423);
\draw [c,line width=0.9] (2.48,11.907) -- (2,11.907);
\draw [anchor= east] (1.844,11.907) node[scale=1.01821, color=c, rotate=0]{1};
\definecolor{c}{rgb}{0,0,1};
\draw [c,line width=0.9] (2.09053,11.9972) -- (2.28837,11.9972) -- (2.48622,11.9972) -- (2.68406,11.9972) -- (2.88191,11.9972) -- (3.07975,11.9972) -- (3.2776,11.9972) -- (3.47544,11.9972) -- (3.67329,11.9972) -- (3.87113,11.9972) --
 (4.06898,11.9972) -- (4.26682,11.9972) -- (4.46467,11.9972) -- (4.66251,11.9972) -- (4.86035,11.9972) -- (5.0582,11.9972) -- (5.25604,11.9972) -- (5.45389,11.9972) -- (5.65173,11.9972) -- (5.84958,11.9972) -- (6.04742,11.9972) -- (6.24527,11.9972)
 -- (6.44311,11.9972) -- (6.64096,11.9972) -- (6.8388,11.9972) -- (7.03665,11.9972) -- (7.23449,11.9972) -- (7.43234,11.9972) -- (7.63018,11.9972) -- (7.82803,11.9972) -- (8.02587,11.9972) -- (8.22371,11.9971) -- (8.42156,11.9971) -- (8.6194,11.9971)
 -- (8.81725,11.9971) -- (9.01509,11.997) -- (9.21294,11.997) -- (9.41078,11.9969) -- (9.60863,11.9967) -- (9.80647,11.9964) -- (10.0043,11.9959) -- (10.2022,11.9952) -- (10.4,11.9941) -- (10.5979,11.9922) -- (10.7957,11.9894) -- (10.9935,11.9848) --
 (11.1914,11.9776) -- (11.3892,11.9663) -- (11.5871,11.9484) -- (11.7849,11.9204) -- (11.9828,11.8768) -- (12.1806,11.8094) -- (12.3785,11.707) -- (12.5763,11.5551) -- (12.7741,11.3375) -- (12.972,11.0399) -- (13.1698,10.655) -- (13.3677,10.1858) --
 (13.5655,9.64421) -- (13.7634,9.04617) -- (13.9612,8.40724) -- (14.1591,7.74017) -- (14.3569,7.05434) -- (14.5547,6.35626) -- (14.7526,5.65028) -- (14.9504,4.93926) -- (15.1483,4.22503) -- (15.3461,3.50876) -- (15.544,2.7912) -- (15.7418,2.07283) --
 (15.9397,1.35394) -- (15.9409,1.34957);
\definecolor{c}{rgb}{0,0,0};
\foreach \P in {(2.09053,11.9972), (2.28837,11.9972), (2.48622,11.9972), (2.68406,11.9972), (2.88191,11.9972), (3.07975,11.9972), (3.2776,11.9972), (3.47544,11.9972), (3.67329,11.9972), (3.87113,11.9972), (4.06898,11.9972), (4.26682,11.9972),
 (4.46467,11.9972), (4.66251,11.9972), (4.86035,11.9972), (5.0582,11.9972), (5.25604,11.9972), (5.45389,11.9972), (5.65173,11.9972), (5.84958,11.9972), (6.04742,11.9972), (6.24527,11.9972), (6.44311,11.9972), (6.64096,11.9972), (6.8388,11.9972),
 (7.03665,11.9972), (7.23449,11.9972), (7.43234,11.9972), (7.63018,11.9972), (7.82803,11.9972), (8.02587,11.9972), (8.22371,11.9971), (8.42156,11.9971), (8.6194,11.9971), (8.81725,11.9971), (9.01509,11.997), (9.21294,11.997), (9.41078,11.9969),
 (9.60863,11.9967), (9.80647,11.9964), (10.0043,11.9959), (10.2022,11.9952), (10.4,11.9941), (10.5979,11.9922), (10.7957,11.9894), (10.9935,11.9848), (11.1914,11.9776), (11.3892,11.9663), (11.5871,11.9484), (11.7849,11.9204), (11.9828,11.8768),
 (12.1806,11.8094), (12.3785,11.707), (12.5763,11.5551), (12.7741,11.3375), (12.972,11.0399), (13.1698,10.655), (13.3677,10.1858), (13.5655,9.64421), (13.7634,9.04617), (13.9612,8.40724), (14.1591,7.74017), (14.3569,7.05434), (14.5547,6.35626),
 (14.7526,5.65028), (14.9504,4.93926), (15.1483,4.22503), (15.3461,3.50876), (15.544,2.7912), (15.7418,2.07283), (15.9397,1.35394)}{\draw[mark options={color=c,fill=c},mark size=2.402402pt,mark=*,mark size=1pt] plot coordinates {\P};}
\draw (8.45788,13.0156) node[scale=1.27276, color=c, rotate=0]{Confronto tra dati e simulazione};
\definecolor{c}{rgb}{1,0,0};
\draw [c,line width=0.9] (2.09053,11.9226) -- (3.03449,11.9226) -- (4.06898,11.9226) -- (5.01294,11.9226) -- (6.04742,11.9226) -- (6.99138,11.907) -- (8.02587,11.907) -- (8.96983,11.9226) -- (10.0043,11.938) -- (10.9483,12.0273) -- (11.6719,12.1461);
\draw [c,line width=0.9] (11.8019,12.1461) -- (11.8922,11.9833);
\draw [c,line width=0.9] (11.8922,11.9833) -- (11.9828,11.7423) -- (12.0878,11.3717) -- (12.3209,10.4585) -- (12.5501,9.48424) -- (12.9267,8.14348) -- (13.5223,6.42639);
\definecolor{c}{rgb}{0.573333,0,1};
\foreach \P in {(2.09053,11.9226), (3.03449,11.9226), (4.06898,11.9226), (5.01294,11.9226), (6.04742,11.9226), (6.99138,11.907), (8.02587,11.907), (8.96983,11.9226), (10.0043,11.938), (10.9483,12.0273)}{\draw[mark options={color=c,fill=c},mark
 size=1.201201pt,mark=o] plot coordinates {\P};}
\foreach \P in {(11.8922,11.9833), (11.9828,11.7423), (12.0878,11.3717), (12.3209,10.4585), (12.5501,9.48424), (12.9267,8.14348), (13.5223,6.42639)}{\draw[mark options={color=c,fill=c},mark size=1.201201pt,mark=o] plot coordinates {\P};}
\end{tikzpicture}

 }%
 \caption{Risposta in frequenza con Spice confrontata coi dati sperimentali (A=1)} 
 \label{gr:sim_a1.tex} 
\end{grafico}

\begin{grafico}
 \centering
 \resizebox{\textwidth}{!}{%
 \begin{tikzpicture}
\pgfdeclareplotmark{cross} {
\pgfpathmoveto{\pgfpoint{-0.3\pgfplotmarksize}{\pgfplotmarksize}}
\pgfpathlineto{\pgfpoint{+0.3\pgfplotmarksize}{\pgfplotmarksize}}
\pgfpathlineto{\pgfpoint{+0.3\pgfplotmarksize}{0.3\pgfplotmarksize}}
\pgfpathlineto{\pgfpoint{+1\pgfplotmarksize}{0.3\pgfplotmarksize}}
\pgfpathlineto{\pgfpoint{+1\pgfplotmarksize}{-0.3\pgfplotmarksize}}
\pgfpathlineto{\pgfpoint{+0.3\pgfplotmarksize}{-0.3\pgfplotmarksize}}
\pgfpathlineto{\pgfpoint{+0.3\pgfplotmarksize}{-1.\pgfplotmarksize}}
\pgfpathlineto{\pgfpoint{-0.3\pgfplotmarksize}{-1.\pgfplotmarksize}}
\pgfpathlineto{\pgfpoint{-0.3\pgfplotmarksize}{-0.3\pgfplotmarksize}}
\pgfpathlineto{\pgfpoint{-1.\pgfplotmarksize}{-0.3\pgfplotmarksize}}
\pgfpathlineto{\pgfpoint{-1.\pgfplotmarksize}{0.3\pgfplotmarksize}}
\pgfpathlineto{\pgfpoint{-0.3\pgfplotmarksize}{0.3\pgfplotmarksize}}
\pgfpathclose
\pgfusepathqstroke
}
\pgfdeclareplotmark{cross*} {
\pgfpathmoveto{\pgfpoint{-0.3\pgfplotmarksize}{\pgfplotmarksize}}
\pgfpathlineto{\pgfpoint{+0.3\pgfplotmarksize}{\pgfplotmarksize}}
\pgfpathlineto{\pgfpoint{+0.3\pgfplotmarksize}{0.3\pgfplotmarksize}}
\pgfpathlineto{\pgfpoint{+1\pgfplotmarksize}{0.3\pgfplotmarksize}}
\pgfpathlineto{\pgfpoint{+1\pgfplotmarksize}{-0.3\pgfplotmarksize}}
\pgfpathlineto{\pgfpoint{+0.3\pgfplotmarksize}{-0.3\pgfplotmarksize}}
\pgfpathlineto{\pgfpoint{+0.3\pgfplotmarksize}{-1.\pgfplotmarksize}}
\pgfpathlineto{\pgfpoint{-0.3\pgfplotmarksize}{-1.\pgfplotmarksize}}
\pgfpathlineto{\pgfpoint{-0.3\pgfplotmarksize}{-0.3\pgfplotmarksize}}
\pgfpathlineto{\pgfpoint{-1.\pgfplotmarksize}{-0.3\pgfplotmarksize}}
\pgfpathlineto{\pgfpoint{-1.\pgfplotmarksize}{0.3\pgfplotmarksize}}
\pgfpathlineto{\pgfpoint{-0.3\pgfplotmarksize}{0.3\pgfplotmarksize}}
\pgfpathclose
\pgfusepathqfillstroke
}
\pgfdeclareplotmark{newstar} {
\pgfpathmoveto{\pgfqpoint{0pt}{\pgfplotmarksize}}
\pgfpathlineto{\pgfqpointpolar{44}{0.5\pgfplotmarksize}}
\pgfpathlineto{\pgfqpointpolar{18}{\pgfplotmarksize}}
\pgfpathlineto{\pgfqpointpolar{-20}{0.5\pgfplotmarksize}}
\pgfpathlineto{\pgfqpointpolar{-54}{\pgfplotmarksize}}
\pgfpathlineto{\pgfqpointpolar{-90}{0.5\pgfplotmarksize}}
\pgfpathlineto{\pgfqpointpolar{234}{\pgfplotmarksize}}
\pgfpathlineto{\pgfqpointpolar{198}{0.5\pgfplotmarksize}}
\pgfpathlineto{\pgfqpointpolar{162}{\pgfplotmarksize}}
\pgfpathlineto{\pgfqpointpolar{134}{0.5\pgfplotmarksize}}
\pgfpathclose
\pgfusepathqstroke
}
\pgfdeclareplotmark{newstar*} {
\pgfpathmoveto{\pgfqpoint{0pt}{\pgfplotmarksize}}
\pgfpathlineto{\pgfqpointpolar{44}{0.5\pgfplotmarksize}}
\pgfpathlineto{\pgfqpointpolar{18}{\pgfplotmarksize}}
\pgfpathlineto{\pgfqpointpolar{-20}{0.5\pgfplotmarksize}}
\pgfpathlineto{\pgfqpointpolar{-54}{\pgfplotmarksize}}
\pgfpathlineto{\pgfqpointpolar{-90}{0.5\pgfplotmarksize}}
\pgfpathlineto{\pgfqpointpolar{234}{\pgfplotmarksize}}
\pgfpathlineto{\pgfqpointpolar{198}{0.5\pgfplotmarksize}}
\pgfpathlineto{\pgfqpointpolar{162}{\pgfplotmarksize}}
\pgfpathlineto{\pgfqpointpolar{134}{0.5\pgfplotmarksize}}
\pgfpathclose
\pgfusepathqfillstroke
}
\definecolor{c}{rgb}{1,1,1};
\draw [color=c, fill=c] (0,0) rectangle (20,13.4957);
\draw [color=c, fill=c] (2,1.34957) rectangle (18,12.1461);
\definecolor{c}{rgb}{0,0,0};
\draw [c,line width=0.9] (2,1.34957) -- (2,12.1461) -- (18,12.1461) -- (18,1.34957) -- (2,1.34957);
\definecolor{c}{rgb}{1,1,1};
\draw [color=c, fill=c] (2,1.34957) rectangle (18,12.1461);
\definecolor{c}{rgb}{0,0,0};
\draw [c,line width=0.9] (2,1.34957) -- (2,12.1461) -- (18,12.1461) -- (18,1.34957) -- (2,1.34957);
\draw [c,line width=0.9] (2,1.34957) -- (18,1.34957);
\draw [c,dotted,line width=0.9] (2.09053,12.1461) -- (2.09053,1.34957);
\draw [c,dotted,line width=0.9] (4.06897,12.1461) -- (4.06897,1.34957);
\draw [c,dotted,line width=0.9] (6.04742,12.1461) -- (6.04742,1.34957);
\draw [c,dotted,line width=0.9] (8.02587,12.1461) -- (8.02587,1.34957);
\draw [c,dotted,line width=0.9] (10.0043,12.1461) -- (10.0043,1.34957);
\draw [c,dotted,line width=0.9] (11.9828,12.1461) -- (11.9828,1.34957);
\draw [c,dotted,line width=0.9] (13.9612,12.1461) -- (13.9612,1.34957);
\draw [c,dotted,line width=0.9] (15.9397,12.1461) -- (15.9397,1.34957);
\draw [c,dotted,line width=0.9] (17.9181,12.1461) -- (17.9181,1.34957);
\draw [c,line width=0.9] (2,1.34957) -- (2,12.1461);
\draw [c,dotted,line width=0.9] (18,1.44637) -- (2,1.44637);
\draw [c,dotted,line width=0.9] (18,1.68731) -- (2,1.68731);
\draw [c,dotted,line width=0.9] (18,1.89601) -- (2,1.89601);
\draw [c,dotted,line width=0.9] (18,2.0801) -- (2,2.0801);
\draw [c,dotted,line width=0.9] (18,2.24478) -- (2,2.24478);
\draw [c,dotted,line width=0.9] (18,3.32814) -- (2,3.32814);
\draw [c,dotted,line width=0.9] (18,3.96186) -- (2,3.96186);
\draw [c,dotted,line width=0.9] (18,4.4115) -- (2,4.4115);
\draw [c,dotted,line width=0.9] (18,4.76027) -- (2,4.76027);
\draw [c,dotted,line width=0.9] (18,5.04523) -- (2,5.04523);
\draw [c,dotted,line width=0.9] (18,5.28616) -- (2,5.28616);
\draw [c,dotted,line width=0.9] (18,5.49486) -- (2,5.49486);
\draw [c,dotted,line width=0.9] (18,5.67895) -- (2,5.67895);
\draw [c,dotted,line width=0.9] (18,5.84363) -- (2,5.84363);
\draw [c,dotted,line width=0.9] (18,6.92699) -- (2,6.92699);
\draw [c,dotted,line width=0.9] (18,7.56072) -- (2,7.56072);
\draw [c,dotted,line width=0.9] (18,8.01035) -- (2,8.01035);
\draw [c,dotted,line width=0.9] (18,8.35912) -- (2,8.35912);
\draw [c,dotted,line width=0.9] (18,8.64408) -- (2,8.64408);
\draw [c,dotted,line width=0.9] (18,8.88501) -- (2,8.88501);
\draw [c,dotted,line width=0.9] (18,9.09372) -- (2,9.09372);
\draw [c,dotted,line width=0.9] (18,9.27781) -- (2,9.27781);
\draw [c,dotted,line width=0.9] (18,9.44248) -- (2,9.44248);
\draw [c,dotted,line width=0.9] (18,10.5258) -- (2,10.5258);
\draw [c,dotted,line width=0.9] (18,11.1596) -- (2,11.1596);
\draw [c,dotted,line width=0.9] (18,11.6092) -- (2,11.6092);
\draw [c,dotted,line width=0.9] (18,11.958) -- (2,11.958);
\draw [c,line width=0.9] (2,1.34957) -- (18,1.34957);
\draw [anchor= east] (18,0.593811) node[scale=1.01821, color=c, rotate=0]{f [Hz]};
\draw [c,line width=0.9] (2.09053,1.67347) -- (2.09053,1.34957);
\draw [anchor=base] (2.09053,0.73889) node[scale=1.01821, color=c, rotate=0]{10};
\draw [c,line width=0.9] (2.6861,1.51152) -- (2.6861,1.34957);
\draw [c,line width=0.9] (3.03449,1.51152) -- (3.03449,1.34957);
\draw [c,line width=0.9] (3.28167,1.51152) -- (3.28167,1.34957);
\draw [c,line width=0.9] (3.4734,1.51152) -- (3.4734,1.34957);
\draw [c,line width=0.9] (3.63006,1.51152) -- (3.63006,1.34957);
\draw [c,line width=0.9] (3.76251,1.51152) -- (3.76251,1.34957);
\draw [c,line width=0.9] (3.87724,1.51152) -- (3.87724,1.34957);
\draw [c,line width=0.9] (3.97845,1.51152) -- (3.97845,1.34957);
\draw [c,line width=0.9] (4.06897,1.67347) -- (4.06897,1.34957);
\draw [anchor=base] (4.06897,0.73889) node[scale=1.01821, color=c, rotate=0]{$10^{2}$};
\draw [c,line width=0.9] (4.66455,1.51152) -- (4.66455,1.34957);
\draw [c,line width=0.9] (5.01293,1.51152) -- (5.01293,1.34957);
\draw [c,line width=0.9] (5.26012,1.51152) -- (5.26012,1.34957);
\draw [c,line width=0.9] (5.45185,1.51152) -- (5.45185,1.34957);
\draw [c,line width=0.9] (5.60851,1.51152) -- (5.60851,1.34957);
\draw [c,line width=0.9] (5.74096,1.51152) -- (5.74096,1.34957);
\draw [c,line width=0.9] (5.85569,1.51152) -- (5.85569,1.34957);
\draw [c,line width=0.9] (5.95689,1.51152) -- (5.95689,1.34957);
\draw [c,line width=0.9] (6.04742,1.67347) -- (6.04742,1.34957);
\draw [anchor=base] (6.04742,0.73889) node[scale=1.01821, color=c, rotate=0]{$10^{3}$};
\draw [c,line width=0.9] (6.64299,1.51152) -- (6.64299,1.34957);
\draw [c,line width=0.9] (6.99138,1.51152) -- (6.99138,1.34957);
\draw [c,line width=0.9] (7.23857,1.51152) -- (7.23857,1.34957);
\draw [c,line width=0.9] (7.4303,1.51152) -- (7.4303,1.34957);
\draw [c,line width=0.9] (7.58695,1.51152) -- (7.58695,1.34957);
\draw [c,line width=0.9] (7.7194,1.51152) -- (7.7194,1.34957);
\draw [c,line width=0.9] (7.83414,1.51152) -- (7.83414,1.34957);
\draw [c,line width=0.9] (7.93534,1.51152) -- (7.93534,1.34957);
\draw [c,line width=0.9] (8.02587,1.67347) -- (8.02587,1.34957);
\draw [anchor=base] (8.02587,0.73889) node[scale=1.01821, color=c, rotate=0]{$10^{4}$};
\draw [c,line width=0.9] (8.62144,1.51152) -- (8.62144,1.34957);
\draw [c,line width=0.9] (8.96983,1.51152) -- (8.96983,1.34957);
\draw [c,line width=0.9] (9.21701,1.51152) -- (9.21701,1.34957);
\draw [c,line width=0.9] (9.40874,1.51152) -- (9.40874,1.34957);
\draw [c,line width=0.9] (9.5654,1.51152) -- (9.5654,1.34957);
\draw [c,line width=0.9] (9.69785,1.51152) -- (9.69785,1.34957);
\draw [c,line width=0.9] (9.81259,1.51152) -- (9.81259,1.34957);
\draw [c,line width=0.9] (9.91379,1.51152) -- (9.91379,1.34957);
\draw [c,line width=0.9] (10.0043,1.67347) -- (10.0043,1.34957);
\draw [anchor=base] (10.0043,0.73889) node[scale=1.01821, color=c, rotate=0]{$10^{5}$};
\draw [c,line width=0.9] (10.5999,1.51152) -- (10.5999,1.34957);
\draw [c,line width=0.9] (10.9483,1.51152) -- (10.9483,1.34957);
\draw [c,line width=0.9] (11.1955,1.51152) -- (11.1955,1.34957);
\draw [c,line width=0.9] (11.3872,1.51152) -- (11.3872,1.34957);
\draw [c,line width=0.9] (11.5438,1.51152) -- (11.5438,1.34957);
\draw [c,line width=0.9] (11.6763,1.51152) -- (11.6763,1.34957);
\draw [c,line width=0.9] (11.791,1.51152) -- (11.791,1.34957);
\draw [c,line width=0.9] (11.8922,1.51152) -- (11.8922,1.34957);
\draw [c,line width=0.9] (11.9828,1.67347) -- (11.9828,1.34957);
\draw [anchor=base] (11.9828,0.73889) node[scale=1.01821, color=c, rotate=0]{$10^{6}$};
\draw [c,line width=0.9] (12.5783,1.51152) -- (12.5783,1.34957);
\draw [c,line width=0.9] (12.9267,1.51152) -- (12.9267,1.34957);
\draw [c,line width=0.9] (13.1739,1.51152) -- (13.1739,1.34957);
\draw [c,line width=0.9] (13.3656,1.51152) -- (13.3656,1.34957);
\draw [c,line width=0.9] (13.5223,1.51152) -- (13.5223,1.34957);
\draw [c,line width=0.9] (13.6547,1.51152) -- (13.6547,1.34957);
\draw [c,line width=0.9] (13.7695,1.51152) -- (13.7695,1.34957);
\draw [c,line width=0.9] (13.8707,1.51152) -- (13.8707,1.34957);
\draw [c,line width=0.9] (13.9612,1.67347) -- (13.9612,1.34957);
\draw [anchor=base] (13.9612,0.73889) node[scale=1.01821, color=c, rotate=0]{$10^{7}$};
\draw [c,line width=0.9] (14.5568,1.51152) -- (14.5568,1.34957);
\draw [c,line width=0.9] (14.9052,1.51152) -- (14.9052,1.34957);
\draw [c,line width=0.9] (15.1524,1.51152) -- (15.1524,1.34957);
\draw [c,line width=0.9] (15.3441,1.51152) -- (15.3441,1.34957);
\draw [c,line width=0.9] (15.5007,1.51152) -- (15.5007,1.34957);
\draw [c,line width=0.9] (15.6332,1.51152) -- (15.6332,1.34957);
\draw [c,line width=0.9] (15.7479,1.51152) -- (15.7479,1.34957);
\draw [c,line width=0.9] (15.8491,1.51152) -- (15.8491,1.34957);
\draw [c,line width=0.9] (15.9397,1.67347) -- (15.9397,1.34957);
\draw [anchor=base] (15.9397,0.73889) node[scale=1.01821, color=c, rotate=0]{$10^{8}$};
\draw [c,line width=0.9] (16.5352,1.51152) -- (16.5352,1.34957);
\draw [c,line width=0.9] (16.8836,1.51152) -- (16.8836,1.34957);
\draw [c,line width=0.9] (17.1308,1.51152) -- (17.1308,1.34957);
\draw [c,line width=0.9] (17.3225,1.51152) -- (17.3225,1.34957);
\draw [c,line width=0.9] (17.4792,1.51152) -- (17.4792,1.34957);
\draw [c,line width=0.9] (17.6116,1.51152) -- (17.6116,1.34957);
\draw [c,line width=0.9] (17.7264,1.51152) -- (17.7264,1.34957);
\draw [c,line width=0.9] (17.8276,1.51152) -- (17.8276,1.34957);
\draw [c,line width=0.9] (17.9181,1.67347) -- (17.9181,1.34957);
\draw [anchor=base] (17.9181,0.73889) node[scale=1.01821, color=c, rotate=0]{$10^{9}$};
\draw [c,line width=0.9] (2,1.34957) -- (2,12.1461);
\draw [anchor= east] (0.88,12.1461) node[scale=1.01821, color=c, rotate=90]{A};
\draw [c,line width=0.9] (2.24,1.44637) -- (2,1.44637);
\draw [c,line width=0.9] (2.24,1.68731) -- (2,1.68731);
\draw [c,line width=0.9] (2.24,1.89601) -- (2,1.89601);
\draw [c,line width=0.9] (2.24,2.0801) -- (2,2.0801);
\draw [c,line width=0.9] (2.48,2.24478) -- (2,2.24478);
\draw [anchor= east] (1.844,2.24478) node[scale=1.01821, color=c, rotate=0]{$10^{-2}$};
\draw [c,line width=0.9] (2.24,3.32814) -- (2,3.32814);
\draw [c,line width=0.9] (2.24,3.96186) -- (2,3.96186);
\draw [c,line width=0.9] (2.24,4.4115) -- (2,4.4115);
\draw [c,line width=0.9] (2.24,4.76027) -- (2,4.76027);
\draw [c,line width=0.9] (2.24,5.04523) -- (2,5.04523);
\draw [c,line width=0.9] (2.24,5.28616) -- (2,5.28616);
\draw [c,line width=0.9] (2.24,5.49486) -- (2,5.49486);
\draw [c,line width=0.9] (2.24,5.67895) -- (2,5.67895);
\draw [c,line width=0.9] (2.48,5.84363) -- (2,5.84363);
\draw [anchor= east] (1.844,5.84363) node[scale=1.01821, color=c, rotate=0]{$10^{-1}$};
\draw [c,line width=0.9] (2.24,6.92699) -- (2,6.92699);
\draw [c,line width=0.9] (2.24,7.56072) -- (2,7.56072);
\draw [c,line width=0.9] (2.24,8.01035) -- (2,8.01035);
\draw [c,line width=0.9] (2.24,8.35912) -- (2,8.35912);
\draw [c,line width=0.9] (2.24,8.64408) -- (2,8.64408);
\draw [c,line width=0.9] (2.24,8.88501) -- (2,8.88501);
\draw [c,line width=0.9] (2.24,9.09372) -- (2,9.09372);
\draw [c,line width=0.9] (2.24,9.27781) -- (2,9.27781);
\draw [c,line width=0.9] (2.48,9.44248) -- (2,9.44248);
\draw [anchor= east] (1.844,9.44248) node[scale=1.01821, color=c, rotate=0]{1};
\draw [c,line width=0.9] (2.24,10.5258) -- (2,10.5258);
\draw [c,line width=0.9] (2.24,11.1596) -- (2,11.1596);
\draw [c,line width=0.9] (2.24,11.6092) -- (2,11.6092);
\draw [c,line width=0.9] (2.24,11.958) -- (2,11.958);
\definecolor{c}{rgb}{0,0,1};
\draw [c,line width=0.9] (2.09053,11.9972) -- (2.28837,11.9972) -- (2.48622,11.9972) -- (2.68406,11.9972) -- (2.88191,11.9972) -- (3.07975,11.9972) -- (3.2776,11.9972) -- (3.47544,11.9972) -- (3.67329,11.9972) -- (3.87113,11.9972) --
 (4.06898,11.9972) -- (4.26682,11.9972) -- (4.46467,11.9972) -- (4.66251,11.9972) -- (4.86035,11.9972) -- (5.0582,11.9972) -- (5.25604,11.9972) -- (5.45389,11.9972) -- (5.65173,11.9972) -- (5.84958,11.9972) -- (6.04742,11.9972) -- (6.24527,11.9972)
 -- (6.44311,11.9972) -- (6.64096,11.9972) -- (6.8388,11.9972) -- (7.03665,11.9971) -- (7.23449,11.9971) -- (7.43234,11.9971) -- (7.63018,11.9971) -- (7.82803,11.997) -- (8.02587,11.997) -- (8.22371,11.9969) -- (8.42156,11.9967) -- (8.6194,11.9964)
 -- (8.81725,11.996) -- (9.01509,11.9953) -- (9.21294,11.9942) -- (9.41078,11.9925) -- (9.60863,11.9898) -- (9.80647,11.9855) -- (10.0043,11.9788) -- (10.2022,11.9682) -- (10.4,11.9517) -- (10.5979,11.9263) -- (10.7957,11.8875) -- (10.9935,11.8297)
 -- (11.1914,11.7458) -- (11.3892,11.6285) -- (11.5871,11.4714) -- (11.7849,11.2714) -- (11.9828,11.0288) -- (12.1806,10.7468) -- (12.3785,10.43) -- (12.5763,10.0817) -- (12.7741,9.703) -- (12.972,9.29213) -- (13.1698,8.84535) -- (13.3677,8.35833) --
 (13.5655,7.82829) -- (13.7634,7.25572) -- (13.9612,6.64474) -- (14.1591,6.00201) -- (14.3569,5.33493) -- (14.5547,4.65027) -- (14.7526,3.95347) -- (14.9504,3.24855) -- (15.1483,2.53828) -- (15.3461,1.82457) -- (15.4774,1.34957);
\definecolor{c}{rgb}{0,0,0};
\foreach \P in {(2.09053,11.9972), (2.28837,11.9972), (2.48622,11.9972), (2.68406,11.9972), (2.88191,11.9972), (3.07975,11.9972), (3.2776,11.9972), (3.47544,11.9972), (3.67329,11.9972), (3.87113,11.9972), (4.06898,11.9972), (4.26682,11.9972),
 (4.46467,11.9972), (4.66251,11.9972), (4.86035,11.9972), (5.0582,11.9972), (5.25604,11.9972), (5.45389,11.9972), (5.65173,11.9972), (5.84958,11.9972), (6.04742,11.9972), (6.24527,11.9972), (6.44311,11.9972), (6.64096,11.9972), (6.8388,11.9972),
 (7.03665,11.9971), (7.23449,11.9971), (7.43234,11.9971), (7.63018,11.9971), (7.82803,11.997), (8.02587,11.997), (8.22371,11.9969), (8.42156,11.9967), (8.6194,11.9964), (8.81725,11.996), (9.01509,11.9953), (9.21294,11.9942), (9.41078,11.9925),
 (9.60863,11.9898), (9.80647,11.9855), (10.0043,11.9788), (10.2022,11.9682), (10.4,11.9517), (10.5979,11.9263), (10.7957,11.8875), (10.9935,11.8297), (11.1914,11.7458), (11.3892,11.6285), (11.5871,11.4714), (11.7849,11.2714), (11.9828,11.0288),
 (12.1806,10.7468), (12.3785,10.43), (12.5763,10.0817), (12.7741,9.703), (12.972,9.29213), (13.1698,8.84535), (13.3677,8.35833), (13.5655,7.82829), (13.7634,7.25572), (13.9612,6.64474), (14.1591,6.00201), (14.3569,5.33493), (14.5547,4.65027),
 (14.7526,3.95347), (14.9504,3.24855), (15.1483,2.53828), (15.3461,1.82457)}{\draw[mark options={color=c,fill=c},mark size=2.402402pt,mark=*,mark size=1pt] plot coordinates {\P};}
\draw (8.45788,13.0156) node[scale=1.52731, color=c, rotate=0]{Confronto tra dati e simulazione};
\definecolor{c}{rgb}{1,0,0};
\draw [c,line width=0.9] (2.09053,11.9296) -- (3.28167,11.9136) -- (4.06898,11.9359) -- (5.01294,11.92) -- (6.04742,11.92) -- (6.99138,11.9136) -- (8.02587,11.9136) -- (8.96983,11.9104) -- (10.0043,11.8942) -- (10.9483,11.7295) -- (11.1955,11.5935)
 -- (11.4126,11.4005) -- (11.5438,11.2408) -- (11.9771,10.5444) -- (12.9267,8.15932) -- (13.3656,6.67298);
\definecolor{c}{rgb}{0.573333,0,1};
\foreach \P in {(2.09053,11.9296), (3.28167,11.9136), (4.06898,11.9359), (5.01294,11.92), (6.04742,11.92), (6.99138,11.9136), (8.02587,11.9136), (8.96983,11.9104), (10.0043,11.8942), (10.9483,11.7295), (11.1955,11.5935), (11.4126,11.4005),
 (11.5438,11.2408), (11.9771,10.5444), (12.9267,8.15932), (13.3656,6.67298)}{\draw[mark options={color=c,fill=c},mark size=1.201201pt,mark=o] plot coordinates {\P};}
\end{tikzpicture}

 }%
 \caption{Risposta in frequenza con Spice confrontata coi dati sperimentali (A=5)} 
 \label{gr:sim_a5.tex} 
\end{grafico}

\begin{grafico}
 \centering
 \resizebox{\textwidth}{!}{%
 \begin{tikzpicture}
\pgfdeclareplotmark{cross} {
\pgfpathmoveto{\pgfpoint{-0.3\pgfplotmarksize}{\pgfplotmarksize}}
\pgfpathlineto{\pgfpoint{+0.3\pgfplotmarksize}{\pgfplotmarksize}}
\pgfpathlineto{\pgfpoint{+0.3\pgfplotmarksize}{0.3\pgfplotmarksize}}
\pgfpathlineto{\pgfpoint{+1\pgfplotmarksize}{0.3\pgfplotmarksize}}
\pgfpathlineto{\pgfpoint{+1\pgfplotmarksize}{-0.3\pgfplotmarksize}}
\pgfpathlineto{\pgfpoint{+0.3\pgfplotmarksize}{-0.3\pgfplotmarksize}}
\pgfpathlineto{\pgfpoint{+0.3\pgfplotmarksize}{-1.\pgfplotmarksize}}
\pgfpathlineto{\pgfpoint{-0.3\pgfplotmarksize}{-1.\pgfplotmarksize}}
\pgfpathlineto{\pgfpoint{-0.3\pgfplotmarksize}{-0.3\pgfplotmarksize}}
\pgfpathlineto{\pgfpoint{-1.\pgfplotmarksize}{-0.3\pgfplotmarksize}}
\pgfpathlineto{\pgfpoint{-1.\pgfplotmarksize}{0.3\pgfplotmarksize}}
\pgfpathlineto{\pgfpoint{-0.3\pgfplotmarksize}{0.3\pgfplotmarksize}}
\pgfpathclose
\pgfusepathqstroke
}
\pgfdeclareplotmark{cross*} {
\pgfpathmoveto{\pgfpoint{-0.3\pgfplotmarksize}{\pgfplotmarksize}}
\pgfpathlineto{\pgfpoint{+0.3\pgfplotmarksize}{\pgfplotmarksize}}
\pgfpathlineto{\pgfpoint{+0.3\pgfplotmarksize}{0.3\pgfplotmarksize}}
\pgfpathlineto{\pgfpoint{+1\pgfplotmarksize}{0.3\pgfplotmarksize}}
\pgfpathlineto{\pgfpoint{+1\pgfplotmarksize}{-0.3\pgfplotmarksize}}
\pgfpathlineto{\pgfpoint{+0.3\pgfplotmarksize}{-0.3\pgfplotmarksize}}
\pgfpathlineto{\pgfpoint{+0.3\pgfplotmarksize}{-1.\pgfplotmarksize}}
\pgfpathlineto{\pgfpoint{-0.3\pgfplotmarksize}{-1.\pgfplotmarksize}}
\pgfpathlineto{\pgfpoint{-0.3\pgfplotmarksize}{-0.3\pgfplotmarksize}}
\pgfpathlineto{\pgfpoint{-1.\pgfplotmarksize}{-0.3\pgfplotmarksize}}
\pgfpathlineto{\pgfpoint{-1.\pgfplotmarksize}{0.3\pgfplotmarksize}}
\pgfpathlineto{\pgfpoint{-0.3\pgfplotmarksize}{0.3\pgfplotmarksize}}
\pgfpathclose
\pgfusepathqfillstroke
}
\pgfdeclareplotmark{newstar} {
\pgfpathmoveto{\pgfqpoint{0pt}{\pgfplotmarksize}}
\pgfpathlineto{\pgfqpointpolar{44}{0.5\pgfplotmarksize}}
\pgfpathlineto{\pgfqpointpolar{18}{\pgfplotmarksize}}
\pgfpathlineto{\pgfqpointpolar{-20}{0.5\pgfplotmarksize}}
\pgfpathlineto{\pgfqpointpolar{-54}{\pgfplotmarksize}}
\pgfpathlineto{\pgfqpointpolar{-90}{0.5\pgfplotmarksize}}
\pgfpathlineto{\pgfqpointpolar{234}{\pgfplotmarksize}}
\pgfpathlineto{\pgfqpointpolar{198}{0.5\pgfplotmarksize}}
\pgfpathlineto{\pgfqpointpolar{162}{\pgfplotmarksize}}
\pgfpathlineto{\pgfqpointpolar{134}{0.5\pgfplotmarksize}}
\pgfpathclose
\pgfusepathqstroke
}
\pgfdeclareplotmark{newstar*} {
\pgfpathmoveto{\pgfqpoint{0pt}{\pgfplotmarksize}}
\pgfpathlineto{\pgfqpointpolar{44}{0.5\pgfplotmarksize}}
\pgfpathlineto{\pgfqpointpolar{18}{\pgfplotmarksize}}
\pgfpathlineto{\pgfqpointpolar{-20}{0.5\pgfplotmarksize}}
\pgfpathlineto{\pgfqpointpolar{-54}{\pgfplotmarksize}}
\pgfpathlineto{\pgfqpointpolar{-90}{0.5\pgfplotmarksize}}
\pgfpathlineto{\pgfqpointpolar{234}{\pgfplotmarksize}}
\pgfpathlineto{\pgfqpointpolar{198}{0.5\pgfplotmarksize}}
\pgfpathlineto{\pgfqpointpolar{162}{\pgfplotmarksize}}
\pgfpathlineto{\pgfqpointpolar{134}{0.5\pgfplotmarksize}}
\pgfpathclose
\pgfusepathqfillstroke
}
\definecolor{c}{rgb}{1,1,1};
\draw [color=c, fill=c] (0,0) rectangle (20,13.4957);
\draw [color=c, fill=c] (2,1.34957) rectangle (18,12.1461);
\definecolor{c}{rgb}{0,0,0};
\draw [c,line width=0.9] (2,1.34957) -- (2,12.1461) -- (18,12.1461) -- (18,1.34957) -- (2,1.34957);
\definecolor{c}{rgb}{1,1,1};
\draw [color=c, fill=c] (2,1.34957) rectangle (18,12.1461);
\definecolor{c}{rgb}{0,0,0};
\draw [c,line width=0.9] (2,1.34957) -- (2,12.1461) -- (18,12.1461) -- (18,1.34957) -- (2,1.34957);
\draw [c,line width=0.9] (2,1.34957) -- (18,1.34957);
\draw [c,dotted,line width=0.9] (2.09053,12.1461) -- (2.09053,1.34957);
\draw [c,dotted,line width=0.9] (4.06897,12.1461) -- (4.06897,1.34957);
\draw [c,dotted,line width=0.9] (6.04742,12.1461) -- (6.04742,1.34957);
\draw [c,dotted,line width=0.9] (8.02587,12.1461) -- (8.02587,1.34957);
\draw [c,dotted,line width=0.9] (10.0043,12.1461) -- (10.0043,1.34957);
\draw [c,dotted,line width=0.9] (11.9828,12.1461) -- (11.9828,1.34957);
\draw [c,dotted,line width=0.9] (13.9612,12.1461) -- (13.9612,1.34957);
\draw [c,dotted,line width=0.9] (15.9397,12.1461) -- (15.9397,1.34957);
\draw [c,dotted,line width=0.9] (17.9181,12.1461) -- (17.9181,1.34957);
\draw [c,line width=0.9] (2,1.34957) -- (2,12.1461);
\draw [c,dotted,line width=0.9] (18,2.18589) -- (2,2.18589);
\draw [c,dotted,line width=0.9] (18,2.81962) -- (2,2.81962);
\draw [c,dotted,line width=0.9] (18,3.26926) -- (2,3.26926);
\draw [c,dotted,line width=0.9] (18,3.61802) -- (2,3.61802);
\draw [c,dotted,line width=0.9] (18,3.90298) -- (2,3.90298);
\draw [c,dotted,line width=0.9] (18,4.14391) -- (2,4.14391);
\draw [c,dotted,line width=0.9] (18,4.35262) -- (2,4.35262);
\draw [c,dotted,line width=0.9] (18,4.53671) -- (2,4.53671);
\draw [c,dotted,line width=0.9] (18,4.70138) -- (2,4.70138);
\draw [c,dotted,line width=0.9] (18,5.78475) -- (2,5.78475);
\draw [c,dotted,line width=0.9] (18,6.41847) -- (2,6.41847);
\draw [c,dotted,line width=0.9] (18,6.86811) -- (2,6.86811);
\draw [c,dotted,line width=0.9] (18,7.21687) -- (2,7.21687);
\draw [c,dotted,line width=0.9] (18,7.50183) -- (2,7.50183);
\draw [c,dotted,line width=0.9] (18,7.74277) -- (2,7.74277);
\draw [c,dotted,line width=0.9] (18,7.95147) -- (2,7.95147);
\draw [c,dotted,line width=0.9] (18,8.13556) -- (2,8.13556);
\draw [c,dotted,line width=0.9] (18,8.30023) -- (2,8.30023);
\draw [c,dotted,line width=0.9] (18,9.3836) -- (2,9.3836);
\draw [c,dotted,line width=0.9] (18,10.0173) -- (2,10.0173);
\draw [c,dotted,line width=0.9] (18,10.467) -- (2,10.467);
\draw [c,dotted,line width=0.9] (18,10.8157) -- (2,10.8157);
\draw [c,dotted,line width=0.9] (18,11.1007) -- (2,11.1007);
\draw [c,dotted,line width=0.9] (18,11.3416) -- (2,11.3416);
\draw [c,dotted,line width=0.9] (18,11.5503) -- (2,11.5503);
\draw [c,dotted,line width=0.9] (18,11.7344) -- (2,11.7344);
\draw [c,dotted,line width=0.9] (18,11.8991) -- (2,11.8991);
\draw [c,line width=0.9] (2,1.34957) -- (18,1.34957);
\draw [anchor= east] (18,0.593811) node[scale=1.01821, color=c, rotate=0]{f [Hz]};
\draw [c,line width=0.9] (2.09053,1.67347) -- (2.09053,1.34957);
\draw [anchor=base] (2.09053,0.73889) node[scale=1.01821, color=c, rotate=0]{10};
\draw [c,line width=0.9] (2.6861,1.51152) -- (2.6861,1.34957);
\draw [c,line width=0.9] (3.03449,1.51152) -- (3.03449,1.34957);
\draw [c,line width=0.9] (3.28167,1.51152) -- (3.28167,1.34957);
\draw [c,line width=0.9] (3.4734,1.51152) -- (3.4734,1.34957);
\draw [c,line width=0.9] (3.63006,1.51152) -- (3.63006,1.34957);
\draw [c,line width=0.9] (3.76251,1.51152) -- (3.76251,1.34957);
\draw [c,line width=0.9] (3.87724,1.51152) -- (3.87724,1.34957);
\draw [c,line width=0.9] (3.97845,1.51152) -- (3.97845,1.34957);
\draw [c,line width=0.9] (4.06897,1.67347) -- (4.06897,1.34957);
\draw [anchor=base] (4.06897,0.73889) node[scale=1.01821, color=c, rotate=0]{$10^{2}$};
\draw [c,line width=0.9] (4.66455,1.51152) -- (4.66455,1.34957);
\draw [c,line width=0.9] (5.01293,1.51152) -- (5.01293,1.34957);
\draw [c,line width=0.9] (5.26012,1.51152) -- (5.26012,1.34957);
\draw [c,line width=0.9] (5.45185,1.51152) -- (5.45185,1.34957);
\draw [c,line width=0.9] (5.60851,1.51152) -- (5.60851,1.34957);
\draw [c,line width=0.9] (5.74096,1.51152) -- (5.74096,1.34957);
\draw [c,line width=0.9] (5.85569,1.51152) -- (5.85569,1.34957);
\draw [c,line width=0.9] (5.95689,1.51152) -- (5.95689,1.34957);
\draw [c,line width=0.9] (6.04742,1.67347) -- (6.04742,1.34957);
\draw [anchor=base] (6.04742,0.73889) node[scale=1.01821, color=c, rotate=0]{$10^{3}$};
\draw [c,line width=0.9] (6.64299,1.51152) -- (6.64299,1.34957);
\draw [c,line width=0.9] (6.99138,1.51152) -- (6.99138,1.34957);
\draw [c,line width=0.9] (7.23857,1.51152) -- (7.23857,1.34957);
\draw [c,line width=0.9] (7.4303,1.51152) -- (7.4303,1.34957);
\draw [c,line width=0.9] (7.58695,1.51152) -- (7.58695,1.34957);
\draw [c,line width=0.9] (7.7194,1.51152) -- (7.7194,1.34957);
\draw [c,line width=0.9] (7.83414,1.51152) -- (7.83414,1.34957);
\draw [c,line width=0.9] (7.93534,1.51152) -- (7.93534,1.34957);
\draw [c,line width=0.9] (8.02587,1.67347) -- (8.02587,1.34957);
\draw [anchor=base] (8.02587,0.73889) node[scale=1.01821, color=c, rotate=0]{$10^{4}$};
\draw [c,line width=0.9] (8.62144,1.51152) -- (8.62144,1.34957);
\draw [c,line width=0.9] (8.96983,1.51152) -- (8.96983,1.34957);
\draw [c,line width=0.9] (9.21701,1.51152) -- (9.21701,1.34957);
\draw [c,line width=0.9] (9.40874,1.51152) -- (9.40874,1.34957);
\draw [c,line width=0.9] (9.5654,1.51152) -- (9.5654,1.34957);
\draw [c,line width=0.9] (9.69785,1.51152) -- (9.69785,1.34957);
\draw [c,line width=0.9] (9.81259,1.51152) -- (9.81259,1.34957);
\draw [c,line width=0.9] (9.91379,1.51152) -- (9.91379,1.34957);
\draw [c,line width=0.9] (10.0043,1.67347) -- (10.0043,1.34957);
\draw [anchor=base] (10.0043,0.73889) node[scale=1.01821, color=c, rotate=0]{$10^{5}$};
\draw [c,line width=0.9] (10.5999,1.51152) -- (10.5999,1.34957);
\draw [c,line width=0.9] (10.9483,1.51152) -- (10.9483,1.34957);
\draw [c,line width=0.9] (11.1955,1.51152) -- (11.1955,1.34957);
\draw [c,line width=0.9] (11.3872,1.51152) -- (11.3872,1.34957);
\draw [c,line width=0.9] (11.5438,1.51152) -- (11.5438,1.34957);
\draw [c,line width=0.9] (11.6763,1.51152) -- (11.6763,1.34957);
\draw [c,line width=0.9] (11.791,1.51152) -- (11.791,1.34957);
\draw [c,line width=0.9] (11.8922,1.51152) -- (11.8922,1.34957);
\draw [c,line width=0.9] (11.9828,1.67347) -- (11.9828,1.34957);
\draw [anchor=base] (11.9828,0.73889) node[scale=1.01821, color=c, rotate=0]{$10^{6}$};
\draw [c,line width=0.9] (12.5783,1.51152) -- (12.5783,1.34957);
\draw [c,line width=0.9] (12.9267,1.51152) -- (12.9267,1.34957);
\draw [c,line width=0.9] (13.1739,1.51152) -- (13.1739,1.34957);
\draw [c,line width=0.9] (13.3656,1.51152) -- (13.3656,1.34957);
\draw [c,line width=0.9] (13.5223,1.51152) -- (13.5223,1.34957);
\draw [c,line width=0.9] (13.6547,1.51152) -- (13.6547,1.34957);
\draw [c,line width=0.9] (13.7695,1.51152) -- (13.7695,1.34957);
\draw [c,line width=0.9] (13.8707,1.51152) -- (13.8707,1.34957);
\draw [c,line width=0.9] (13.9612,1.67347) -- (13.9612,1.34957);
\draw [anchor=base] (13.9612,0.73889) node[scale=1.01821, color=c, rotate=0]{$10^{7}$};
\draw [c,line width=0.9] (14.5568,1.51152) -- (14.5568,1.34957);
\draw [c,line width=0.9] (14.9052,1.51152) -- (14.9052,1.34957);
\draw [c,line width=0.9] (15.1524,1.51152) -- (15.1524,1.34957);
\draw [c,line width=0.9] (15.3441,1.51152) -- (15.3441,1.34957);
\draw [c,line width=0.9] (15.5007,1.51152) -- (15.5007,1.34957);
\draw [c,line width=0.9] (15.6332,1.51152) -- (15.6332,1.34957);
\draw [c,line width=0.9] (15.7479,1.51152) -- (15.7479,1.34957);
\draw [c,line width=0.9] (15.8491,1.51152) -- (15.8491,1.34957);
\draw [c,line width=0.9] (15.9397,1.67347) -- (15.9397,1.34957);
\draw [anchor=base] (15.9397,0.73889) node[scale=1.01821, color=c, rotate=0]{$10^{8}$};
\draw [c,line width=0.9] (16.5352,1.51152) -- (16.5352,1.34957);
\draw [c,line width=0.9] (16.8836,1.51152) -- (16.8836,1.34957);
\draw [c,line width=0.9] (17.1308,1.51152) -- (17.1308,1.34957);
\draw [c,line width=0.9] (17.3225,1.51152) -- (17.3225,1.34957);
\draw [c,line width=0.9] (17.4792,1.51152) -- (17.4792,1.34957);
\draw [c,line width=0.9] (17.6116,1.51152) -- (17.6116,1.34957);
\draw [c,line width=0.9] (17.7264,1.51152) -- (17.7264,1.34957);
\draw [c,line width=0.9] (17.8276,1.51152) -- (17.8276,1.34957);
\draw [c,line width=0.9] (17.9181,1.67347) -- (17.9181,1.34957);
\draw [anchor=base] (17.9181,0.73889) node[scale=1.01821, color=c, rotate=0]{$10^{9}$};
\draw [c,line width=0.9] (2,1.34957) -- (2,12.1461);
\draw [anchor= east] (0.88,12.1461) node[scale=1.01821, color=c, rotate=90]{A};
\draw [c,line width=0.9] (2.24,2.18589) -- (2,2.18589);
\draw [c,line width=0.9] (2.24,2.81962) -- (2,2.81962);
\draw [c,line width=0.9] (2.24,3.26926) -- (2,3.26926);
\draw [c,line width=0.9] (2.24,3.61802) -- (2,3.61802);
\draw [c,line width=0.9] (2.24,3.90298) -- (2,3.90298);
\draw [c,line width=0.9] (2.24,4.14391) -- (2,4.14391);
\draw [c,line width=0.9] (2.24,4.35262) -- (2,4.35262);
\draw [c,line width=0.9] (2.24,4.53671) -- (2,4.53671);
\draw [c,line width=0.9] (2.48,4.70138) -- (2,4.70138);
\draw [anchor= east] (1.844,4.70138) node[scale=1.01821, color=c, rotate=0]{$10^{-1}$};
\draw [c,line width=0.9] (2.24,5.78475) -- (2,5.78475);
\draw [c,line width=0.9] (2.24,6.41847) -- (2,6.41847);
\draw [c,line width=0.9] (2.24,6.86811) -- (2,6.86811);
\draw [c,line width=0.9] (2.24,7.21687) -- (2,7.21687);
\draw [c,line width=0.9] (2.24,7.50183) -- (2,7.50183);
\draw [c,line width=0.9] (2.24,7.74277) -- (2,7.74277);
\draw [c,line width=0.9] (2.24,7.95147) -- (2,7.95147);
\draw [c,line width=0.9] (2.24,8.13556) -- (2,8.13556);
\draw [c,line width=0.9] (2.48,8.30023) -- (2,8.30023);
\draw [anchor= east] (1.844,8.30023) node[scale=1.01821, color=c, rotate=0]{1};
\draw [c,line width=0.9] (2.24,9.3836) -- (2,9.3836);
\draw [c,line width=0.9] (2.24,10.0173) -- (2,10.0173);
\draw [c,line width=0.9] (2.24,10.467) -- (2,10.467);
\draw [c,line width=0.9] (2.24,10.8157) -- (2,10.8157);
\draw [c,line width=0.9] (2.24,11.1007) -- (2,11.1007);
\draw [c,line width=0.9] (2.24,11.3416) -- (2,11.3416);
\draw [c,line width=0.9] (2.24,11.5503) -- (2,11.5503);
\draw [c,line width=0.9] (2.24,11.7344) -- (2,11.7344);
\draw [c,line width=0.9] (2.48,11.8991) -- (2,11.8991);
\draw [anchor= east] (1.844,11.8991) node[scale=1.01821, color=c, rotate=0]{10};
\definecolor{c}{rgb}{0,0,1};
\draw [c,line width=0.9] (2.09053,11.9972) -- (2.28837,11.9972) -- (2.48622,11.9972) -- (2.68406,11.9972) -- (2.88191,11.9972) -- (3.07975,11.9972) -- (3.2776,11.9972) -- (3.47544,11.9972) -- (3.67329,11.9972) -- (3.87113,11.9972) --
 (4.06898,11.9972) -- (4.26682,11.9972) -- (4.46467,11.9972) -- (4.66251,11.9972) -- (4.86035,11.9972) -- (5.0582,11.9972) -- (5.25604,11.9972) -- (5.45389,11.9972) -- (5.65173,11.9972) -- (5.84958,11.9972) -- (6.04742,11.9972) -- (6.24527,11.9972)
 -- (6.44311,11.9971) -- (6.64096,11.9971) -- (6.8388,11.9971) -- (7.03665,11.9971) -- (7.23449,11.9971) -- (7.43234,11.997) -- (7.63018,11.9969) -- (7.82803,11.9967) -- (8.02587,11.9964) -- (8.22371,11.996) -- (8.42156,11.9953) -- (8.6194,11.9943)
 -- (8.81725,11.9926) -- (9.01509,11.9899) -- (9.21294,11.9857) -- (9.41078,11.9791) -- (9.60863,11.9687) -- (9.80647,11.9525) -- (10.0043,11.9275) -- (10.2022,11.8894) -- (10.4,11.8326) -- (10.5979,11.7502) -- (10.7957,11.635) -- (10.9935,11.4812)
 -- (11.1914,11.286) -- (11.3892,11.0504) -- (11.5871,10.7791) -- (11.7849,10.4783) -- (11.9828,10.154) -- (12.1806,9.81074) -- (12.3785,9.45097) -- (12.5763,9.07468) -- (12.7741,8.67952) -- (12.972,8.26097) -- (13.1698,7.81301) -- (13.3677,7.32939)
 -- (13.5655,6.8055) -- (13.7634,6.24013) -- (13.9612,5.63606) -- (14.1591,4.99912) -- (14.3569,4.33649) -- (14.5547,3.65504) -- (14.7526,2.96044) -- (14.9504,2.25697) -- (15.1483,1.54767) -- (15.2032,1.34957);
\definecolor{c}{rgb}{0,0,0};
\foreach \P in {(2.09053,11.9972), (2.28837,11.9972), (2.48622,11.9972), (2.68406,11.9972), (2.88191,11.9972), (3.07975,11.9972), (3.2776,11.9972), (3.47544,11.9972), (3.67329,11.9972), (3.87113,11.9972), (4.06898,11.9972), (4.26682,11.9972),
 (4.46467,11.9972), (4.66251,11.9972), (4.86035,11.9972), (5.0582,11.9972), (5.25604,11.9972), (5.45389,11.9972), (5.65173,11.9972), (5.84958,11.9972), (6.04742,11.9972), (6.24527,11.9972), (6.44311,11.9971), (6.64096,11.9971), (6.8388,11.9971),
 (7.03665,11.9971), (7.23449,11.9971), (7.43234,11.997), (7.63018,11.9969), (7.82803,11.9967), (8.02587,11.9964), (8.22371,11.996), (8.42156,11.9953), (8.6194,11.9943), (8.81725,11.9926), (9.01509,11.9899), (9.21294,11.9857), (9.41078,11.9791),
 (9.60863,11.9687), (9.80647,11.9525), (10.0043,11.9275), (10.2022,11.8894), (10.4,11.8326), (10.5979,11.7502), (10.7957,11.635), (10.9935,11.4812), (11.1914,11.286), (11.3892,11.0504), (11.5871,10.7791), (11.7849,10.4783), (11.9828,10.154),
 (12.1806,9.81074), (12.3785,9.45097), (12.5763,9.07468), (12.7741,8.67952), (12.972,8.26097), (13.1698,7.81301), (13.3677,7.32939), (13.5655,6.8055), (13.7634,6.24013), (13.9612,5.63606), (14.1591,4.99912), (14.3569,4.33649), (14.5547,3.65504),
 (14.7526,2.96044), (14.9504,2.25697), (15.1483,1.54767)}{\draw[mark options={color=c,fill=c},mark size=2.402402pt,mark=*,mark size=1pt] plot coordinates {\P};}
\draw (8.45788,13.0156) node[scale=1.27276, color=c, rotate=0]{Confronto tra dati e simulazione};
\definecolor{c}{rgb}{1,0,0};
\draw [c,line width=0.9] (2.06304,11.9484) -- (3.46705,11.9198) -- (4.06877,11.9198) -- (5.44413,11.9484) -- (6.04585,11.9198) -- (7.4212,11.9198) -- (9.39828,11.8052) -- (10,11.7765) -- (10.5731,11.4613) -- (10.6304,11.3754) -- (10.659,11.3754) --
 (10.659,11.3754) -- (10.7163,11.3467) -- (10.7736,11.2607) -- (10.9456,11.0888) -- (11.1748,10.7163) -- (11.3754,10.3438) -- (11.9771,9.02579) -- (13.3524,4.87106);
\definecolor{c}{rgb}{0.573333,0,1};
\foreach \P in {(2.06304,11.9484), (3.46705,11.9198), (4.06877,11.9198), (5.44413,11.9484), (6.04585,11.9198), (7.4212,11.9198), (9.39828,11.8052), (10,11.7765), (10.5731,11.4613), (10.6304,11.3754), (10.659,11.3754), (10.659,11.3754),
 (10.7163,11.3467), (10.7736,11.2607), (10.9456,11.0888), (11.1748,10.7163), (11.3754,10.3438), (11.9771,9.02579), (13.3524,4.87106)}{\draw[mark options={color=c,fill=c},mark size=1.201201pt,mark=o] plot coordinates {\P};}
\end{tikzpicture}

 }%
 \caption{Risposta in frequenza con Spice confrontata coi dati sperimentali (A=10)} 
 \label{gr:sim_a10.tex} 
\end{grafico}

Queste differenze si ripercuotono sul Gain-Bandwidth Product. Infatti è più basso di quello previsto nel datasheet fornito dal 
produttore ( $3 \cdot 10^{6}$), differenza visibile soprattutto con amplificazione A=1.
Leggendo il datasheet della Texas Instrument è emerso che una possibile giustificazione a questa discrepanza è che non sono state
seguite alcune linee guida indicate. Non sono stati posti, ad esempio, i condensatori vicino all'amplificatore operazionale,
accorgimento che avrebbe ridotto l'induttanza del circuito di alimentazione ad alte frequenze.
Inoltre l'uso di una breadboard ha introdotto numerose capacità e induttanze parassite, che hanno interferito nella delicata
misura di $f_b$.




	
	%\end{multicols}
\newpage
\section{Codice}
} 	%Fine del grigio e di lsstyle

	\'E presentata qua la parte fondamentale del codice in c++ usato per i calcoli numerici. Inoltre è stato usato per i calcoli Mathematica.
Ma non credo abbia senso metterlo, alla fine ce l'hanno dato loro il codice...

% \begin{lstlisting}[language=C++]
% int main() {}
% \end{lstlisting}
%\lstinputlisting[language=C++]{../src/opamp_p1/Gain.cpp}
	
%\subsection{Esempio immagini}
%\begin{figure}[p]
% \centering
% \includegraphics[width=0.8\textwidth]{spazio1}
% \caption{Spazio!}
% \label{fig:spazio1}
%\end{figure}



\end{document}
