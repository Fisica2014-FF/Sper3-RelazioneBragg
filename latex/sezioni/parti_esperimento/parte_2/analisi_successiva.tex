E' stata implementata la macro \textit{AnaBragg.C} per misurare la larghezza temporale dei segnali. Sono stati presi come riferimenti temporali i valori per i quali il segnale
passa per il (EEEEEE BOH 40 0 45 per cento) del valore massimo. Utilizzando la misura della larghezza temporale dei segnali a 400 $mb$ (NON SO, ANCHE A 380 ALLA FINE?) è stata calcolata la velocità di drift, 
tramite la formula per il calcolo della velocità:

$$ v=\frac{lunghezza camera}{range temporale} $$

La lunghezza della camera di Bragg è 120 mm.
\begin{tabella}
 \centering
 
\begin{center}
\begin{tabulary}{\textwidth}{CCC}
\toprule
 p($mb$)& range $\pm$ $\sigma_{range}$ ($mm$) &  width $\pm$ $\sigma_{width}$($\mu$ s)\\
400&	113.3$\pm$	0.7&	69.52$\pm$	0.04\\
450&	103.2$\pm$	0.6&	63.33$\pm$	0.04\\
500&	94.0$\pm$	0.6&	57.67$\pm$	0.04\\
550&	84.6$\pm$	0.5&	51.93$\pm$	0.03\\
600&	72.5$\pm$	0.4&	44.46$\pm$	0.02\\
650&	66.9$\pm$	0.4&	41.06$\pm$	0.05\\
\bottomrule
\end{tabulary}
\end{center}



 
 \caption{Tabella range spaziali e width temporali}
 \label{tab:range_picco1.tex}
\end{tabella}

\begin{tabella}
 \centering
 
\begin{center}
\begin{tabulary}{\textwidth}{CCC}
\toprule

 p($mb$)& range $\pm$ $\sigma_{range}$ ($mm$) &  width $\pm$ $\sigma_{width}$($\mu$ s)\\
400&	124.0$\pm$  0.8&	76.05$\pm$	0.06\\
450&	111.8$\pm$	0.7&	68.56$\pm$	0.05\\
500&	102.2$\pm$	0.6&	62.68$\pm$	0.05\\
550&	91.7$\pm$	0.6&	56.28$\pm$	0.03\\
600&	78.3$\pm$	0.5&	48.02$\pm$	0.03\\



 \bottomrule
\end{tabulary}
\end{center}
 
 \caption{Tabella range spaziali e width temporali}
 \label{tab:range_picco2.tex}
\end{tabella}

\begin{tabella}
 \centering
 
\begin{center}
\begin{tabulary}{\textwidth}{CCC}
\toprule
p($mb$)& range $\pm$ $\sigma_{range}$ ($mm$) &  width $\pm$ $\sigma_{width}$($\mu$ s)\\
450&	119.6$\pm$	0.7&	73.37$\pm$	0.05\\
500&	109.8$\pm$	0.8&	67.35$\pm$	0.05\\
550&	98.7$\pm$	0.7&	60.56$\pm$	0.05\\
600&	84.0$\pm$	0.6&	51.56$\pm$	0.04\\
\bottomrule
\end{tabulary}
\end{center}
 
 \caption{Tabella range spaziali e width temporali}
 \label{tab:range_picco3.tex}
\end{tabella}

Infine, ad alte pressioni, è stato verificato che il range spaziale fosse inversamente proporzionale alla pressione.