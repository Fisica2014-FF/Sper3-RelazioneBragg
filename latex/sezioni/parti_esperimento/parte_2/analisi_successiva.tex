E' stata implementata la macro \textit{AnaBragg.C} per misurare la larghezza temporale dei segnali. Sono stati presi come riferimenti temporali i valori per i quali il segnale
passa per il (EEEEEE BOH 40 0 45 per cento) del valore massimo. Utilizzando la misura della larghezza temporale dei segnali a 400 $mb$ (NON SO, ANCHE A 380 ALLA FINE?) è stata calcolata la velocità di drift, 
tramite la formula per il calcolo della velocità:

$$ v=\frac{lunghezza camera}{range temporale} $$

La lunghezza della camera di Bragg è 120 mm.

Infine, ad alte pressioni, è stato verificato che il range spaziale fosse inversamente proporzionale alla pressione.