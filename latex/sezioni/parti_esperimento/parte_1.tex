\subsection{Amplificatore invertente}
Schema amplificatore invertente:
%inserire circuito
Le resistenze sono state scelte in modo da avere guadagno $A=-10 \frac{V}{V}$\\
$R_1=9.85 \pm 0.05\,k\Omega $\\ %mettere errore
$R_2=101.3 \pm 0.6\,k\Omega$\\ %mettere errore
$R_3=56.0 \pm 0.3\,\Omega$\\ %mettere errore

Per il calcolo degli errori sul valore delle resistenze, lette sull'Agilent U1232A, è stata utilizzata la seguente formula:
$$\sigma_\textrm{tot}=\sqrt{ \sigma^{2} _\textrm{\%} + \sigma^{2} _\textrm{dgt}}$$
Per il calcolo delle $\sigma_\textrm{tot}$ è stato cercato del datasheet dello strumento, l'errore percentuale e di digit
corrispondente al fondo scala utilizzato.

\subsubsection{Calcolo amplificazione}
La relazione tra le resistenze, affichè soddisfino la richiesta A=10 è la seguente:

$$\frac{V_1-V_n}{R_1}=\frac{V_n-V_0}{R_2}$$
$$V_n=0$$
$$\frac{V_1}{R_1}=\frac{-V_0}{R_2}$$
$$V_0=-\frac{R_2}{R_1} \cdot V_1$$
Da cui si ricava la relazione per il calcolo di A.

\subsubsection{Analisi}
La stima di A teorica, a partire dalle resistenze misurate è:\\
$A_\textrm{teorica}=10.28 \pm 0.08$ %inserire valore teorico

Le misure sono state fatte applicando una tensione sinusoidale di frequenza $ f=1 \,kHz$, variando l'ampiezza tra 
$0.2 V_\textrm{pp}$ e $4 V_\textrm{pp}$.

In seguito è stato fatto il grafico della curva di trasferimento di un amplificatore invertente.

\begin{grafico}
 \centering
 \resizebox{\textwidth}{!}{%
 \input{../grafici/risultati/amp_inv.tex}
 }%
 \caption{Curva di trasferimento di un amplificatore invertente} 
 \label{gr:amp_inv.tex} 
\end{grafico}


In seguito sono presentati i dati del  \autoref{gr:amp_inv.tex} acquisiti in laboratorio, con i rispettivi errori:
\begin{tabella}
 \centering
 \input{tabelle/tab_inv.tex} 
 \caption{Dati curva di trasferimento}
 \label{tab:tab_inv.tex}
\end{tabella}

Per il calcolo degli errori sui valori di $V_\textrm{in}$ e $V_\textrm{out}$ letti sull'oscilloscopio, è stata utilizzata la seguente formula:
$$\sigma_\textrm{tot}=\sqrt{ (0.02\cdot V_\textrm{letto})^2 + (0.06 \cdot V_\textrm{div})^2}$$

E' stata fatta l'interpolazione lineare dei punti nel \autoref{gr:amp_inv.tex} pesata dei punti compresi tra 0 e 1.5 V.\\
$q = 0.02 \pm 0.03 \,V$\\
$m = -10.0 \pm 0.1 \, \frac{V}{V}$




\subsection{Amplificatore non invertente}

Schema amplificatore non invertente:
%inserire circuito
Le resistenze sono state scelte in modo da avere guadagno $A=10 \frac{V}{V}$\\
$R_\textrm{1,up}=9.91 \pm0.05 \,k\Omega $\\ %mettere errore
$R_\textrm{1,down}=9.85 \pm 0.05\,k\Omega$\\ %mettere errore
$R_\textrm{2,up}=99.7 \pm 0.6\,k\Omega$\\ %mettere errore
$R_\textrm{2,down}=101.3 \pm 0.6\,k\Omega$\\
$R_4=56.0 \pm 0.3\,\Omega$

\subsubsection{Calcolo amplificazione}
La relazione tra le resistenze, affichè soddisfino la richiesta A=10 è la seguente:
Nell'ingresso non invertente:
$$\frac{V_1-V_p}{R_\textrm{1down}}=\frac{V_p}{R_\textrm{2down}}$$
$$\frac{V_1}{R_\textrm{1down}}=\frac{V_p}{R_\textrm{1down}}+\frac{V_p}{R_\textrm{2down}}=V_p \left(\frac{1}{R_\textrm{1down}}+\frac{1}{R_\textrm{2down}}\right)$$

Nell'ingresso invertente:
$$\frac{V_0-V_n}{R_\textrm{2up}}=\frac{V_n}{R_\textrm{1up}}$$
$$\frac{V_0}{R_\textrm{2up}}=\frac{V_n}{R_\textrm{1up}}+\frac{V_n}{R_\textrm{2up}}=V_n \left(\frac{1}{R_\textrm{1up}}+\frac{1}{R_\textrm{2up}}\right)$$

Poichè 
$$V_p=V_n$$

$$\frac{V_1}{R_\textrm{1down}} \frac{1}{\left(\frac{1}{R_\textrm{1down}}+\frac{1}{R_\textrm{2down} }\right)}=\frac{V_0}{R_\textrm{2up}} \frac{1}{\left(\frac{1}{R_\textrm{1up}}+\frac{1}{R_\textrm{2up}} \right)}$$
$$V_0=\frac{R_\textrm{2up}}{R_\textrm{1down}} \cdot V_1 \frac{\left(\frac{1}{R_\textrm{1up}}+\frac{1}{R_\textrm{2up} }\right)}{\left(\frac{1}{R_\textrm{1down}}+\frac{1}{R_\textrm{2down}} \right)}$$
Da cui si ricava la relazione per il calcolo di A.
Se poi si assume che $R_\textrm{1down}=R_\textrm{1up}$ e $R_\textrm{2down}=R_\textrm{2up}$, la relazione si semplifica a 
$$V_0=\frac{R_\textrm{2up}}{R_\textrm{1down}} \cdot V_1$$
\subsubsection{Analisi}
La stima di A teorica, a partire dalle resistenze misurate è:\\
$A_\textrm{teorica}=10.08 \pm 0.07$ %inserire valore teorico

Le misure sono state fatte applicando una tensione sinusoidale di frequenza $ f=1 \,kHz$, variando l'ampiezza tra 
$0.2 V_\textrm{pp}$ e $4 V_\textrm{pp}$.

In seguito è stato fatto il grafico della curva di trasferimento di un amplificatore non invertente.

\begin{grafico} 
 \centering
 \resizebox{\textwidth}{!}{%
 \begin{tikzpicture}
\pgfdeclareplotmark{cross} {
\pgfpathmoveto{\pgfpoint{-0.3\pgfplotmarksize}{\pgfplotmarksize}}
\pgfpathlineto{\pgfpoint{+0.3\pgfplotmarksize}{\pgfplotmarksize}}
\pgfpathlineto{\pgfpoint{+0.3\pgfplotmarksize}{0.3\pgfplotmarksize}}
\pgfpathlineto{\pgfpoint{+1\pgfplotmarksize}{0.3\pgfplotmarksize}}
\pgfpathlineto{\pgfpoint{+1\pgfplotmarksize}{-0.3\pgfplotmarksize}}
\pgfpathlineto{\pgfpoint{+0.3\pgfplotmarksize}{-0.3\pgfplotmarksize}}
\pgfpathlineto{\pgfpoint{+0.3\pgfplotmarksize}{-1.\pgfplotmarksize}}
\pgfpathlineto{\pgfpoint{-0.3\pgfplotmarksize}{-1.\pgfplotmarksize}}
\pgfpathlineto{\pgfpoint{-0.3\pgfplotmarksize}{-0.3\pgfplotmarksize}}
\pgfpathlineto{\pgfpoint{-1.\pgfplotmarksize}{-0.3\pgfplotmarksize}}
\pgfpathlineto{\pgfpoint{-1.\pgfplotmarksize}{0.3\pgfplotmarksize}}
\pgfpathlineto{\pgfpoint{-0.3\pgfplotmarksize}{0.3\pgfplotmarksize}}
\pgfpathclose
\pgfusepathqstroke
}
\pgfdeclareplotmark{cross*} {
\pgfpathmoveto{\pgfpoint{-0.3\pgfplotmarksize}{\pgfplotmarksize}}
\pgfpathlineto{\pgfpoint{+0.3\pgfplotmarksize}{\pgfplotmarksize}}
\pgfpathlineto{\pgfpoint{+0.3\pgfplotmarksize}{0.3\pgfplotmarksize}}
\pgfpathlineto{\pgfpoint{+1\pgfplotmarksize}{0.3\pgfplotmarksize}}
\pgfpathlineto{\pgfpoint{+1\pgfplotmarksize}{-0.3\pgfplotmarksize}}
\pgfpathlineto{\pgfpoint{+0.3\pgfplotmarksize}{-0.3\pgfplotmarksize}}
\pgfpathlineto{\pgfpoint{+0.3\pgfplotmarksize}{-1.\pgfplotmarksize}}
\pgfpathlineto{\pgfpoint{-0.3\pgfplotmarksize}{-1.\pgfplotmarksize}}
\pgfpathlineto{\pgfpoint{-0.3\pgfplotmarksize}{-0.3\pgfplotmarksize}}
\pgfpathlineto{\pgfpoint{-1.\pgfplotmarksize}{-0.3\pgfplotmarksize}}
\pgfpathlineto{\pgfpoint{-1.\pgfplotmarksize}{0.3\pgfplotmarksize}}
\pgfpathlineto{\pgfpoint{-0.3\pgfplotmarksize}{0.3\pgfplotmarksize}}
\pgfpathclose
\pgfusepathqfillstroke
}
\pgfdeclareplotmark{newstar} {
\pgfpathmoveto{\pgfqpoint{0pt}{\pgfplotmarksize}}
\pgfpathlineto{\pgfqpointpolar{44}{0.5\pgfplotmarksize}}
\pgfpathlineto{\pgfqpointpolar{18}{\pgfplotmarksize}}
\pgfpathlineto{\pgfqpointpolar{-20}{0.5\pgfplotmarksize}}
\pgfpathlineto{\pgfqpointpolar{-54}{\pgfplotmarksize}}
\pgfpathlineto{\pgfqpointpolar{-90}{0.5\pgfplotmarksize}}
\pgfpathlineto{\pgfqpointpolar{234}{\pgfplotmarksize}}
\pgfpathlineto{\pgfqpointpolar{198}{0.5\pgfplotmarksize}}
\pgfpathlineto{\pgfqpointpolar{162}{\pgfplotmarksize}}
\pgfpathlineto{\pgfqpointpolar{134}{0.5\pgfplotmarksize}}
\pgfpathclose
\pgfusepathqstroke
}
\pgfdeclareplotmark{newstar*} {
\pgfpathmoveto{\pgfqpoint{0pt}{\pgfplotmarksize}}
\pgfpathlineto{\pgfqpointpolar{44}{0.5\pgfplotmarksize}}
\pgfpathlineto{\pgfqpointpolar{18}{\pgfplotmarksize}}
\pgfpathlineto{\pgfqpointpolar{-20}{0.5\pgfplotmarksize}}
\pgfpathlineto{\pgfqpointpolar{-54}{\pgfplotmarksize}}
\pgfpathlineto{\pgfqpointpolar{-90}{0.5\pgfplotmarksize}}
\pgfpathlineto{\pgfqpointpolar{234}{\pgfplotmarksize}}
\pgfpathlineto{\pgfqpointpolar{198}{0.5\pgfplotmarksize}}
\pgfpathlineto{\pgfqpointpolar{162}{\pgfplotmarksize}}
\pgfpathlineto{\pgfqpointpolar{134}{0.5\pgfplotmarksize}}
\pgfpathclose
\pgfusepathqfillstroke
}
\definecolor{c}{rgb}{1,1,1};
\draw [color=c, fill=c] (0,0) rectangle (20,13.5632);
\draw [color=c, fill=c] (2,1.35632) rectangle (18,12.2069);
\definecolor{c}{rgb}{0,0,0};
\draw [c,line width=0.9] (2,1.35632) -- (2,12.2069) -- (18,12.2069) -- (18,1.35632) -- (2,1.35632);
\definecolor{c}{rgb}{1,1,1};
\draw [color=c, fill=c] (2,1.35632) rectangle (18,12.2069);
\definecolor{c}{rgb}{0,0,0};
\draw [c,line width=0.9] (2,1.35632) -- (2,12.2069) -- (18,12.2069) -- (18,1.35632) -- (2,1.35632);
\draw [c,line width=0.9] (2,1.35632) -- (18,1.35632);
\draw [c,dotted,line width=0.9] (3.7792,12.2069) -- (3.7792,1.35632);
\draw [c,dotted,line width=0.9] (6.85062,12.2069) -- (6.85062,1.35632);
\draw [c,dotted,line width=0.9] (9.92204,12.2069) -- (9.92204,1.35632);
\draw [c,dotted,line width=0.9] (12.9935,12.2069) -- (12.9935,1.35632);
\draw [c,dotted,line width=0.9] (16.0649,12.2069) -- (16.0649,1.35632);
\draw [c,dotted,line width=0.9] (3.7792,12.2069) -- (3.7792,1.35632);
\draw [c,dotted,line width=0.9] (16.0649,12.2069) -- (16.0649,1.35632);
\draw [c,line width=0.9] (2,1.35632) -- (2,12.2069);
\draw [c,dotted,line width=0.9] (18,2.19282) -- (2,2.19282);
\draw [c,dotted,line width=0.9] (18,3.69696) -- (2,3.69696);
\draw [c,dotted,line width=0.9] (18,5.2011) -- (2,5.2011);
\draw [c,dotted,line width=0.9] (18,6.70524) -- (2,6.70524);
\draw [c,dotted,line width=0.9] (18,8.20938) -- (2,8.20938);
\draw [c,dotted,line width=0.9] (18,9.71352) -- (2,9.71352);
\draw [c,dotted,line width=0.9] (18,11.2177) -- (2,11.2177);
\draw [c,dotted,line width=0.9] (18,2.19282) -- (2,2.19282);
\draw [c,dotted,line width=0.9] (18,11.2177) -- (2,11.2177);
\draw [c,line width=0.9] (2,1.35632) -- (18,1.35632);
\draw [anchor= east] (18,0.596782) node[scale=1.08496, color=c, rotate=0]{$V_{out} [V]$};
\draw [c,line width=0.9] (3.7792,1.68184) -- (3.7792,1.35632);
\draw [c,line width=0.9] (4.39348,1.51908) -- (4.39348,1.35632);
\draw [c,line width=0.9] (5.00777,1.51908) -- (5.00777,1.35632);
\draw [c,line width=0.9] (5.62205,1.51908) -- (5.62205,1.35632);
\draw [c,line width=0.9] (6.23634,1.51908) -- (6.23634,1.35632);
\draw [c,line width=0.9] (6.85062,1.68184) -- (6.85062,1.35632);
\draw [c,line width=0.9] (7.46491,1.51908) -- (7.46491,1.35632);
\draw [c,line width=0.9] (8.07919,1.51908) -- (8.07919,1.35632);
\draw [c,line width=0.9] (8.69348,1.51908) -- (8.69348,1.35632);
\draw [c,line width=0.9] (9.30776,1.51908) -- (9.30776,1.35632);
\draw [c,line width=0.9] (9.92204,1.68184) -- (9.92204,1.35632);
\draw [c,line width=0.9] (10.5363,1.51908) -- (10.5363,1.35632);
\draw [c,line width=0.9] (11.1506,1.51908) -- (11.1506,1.35632);
\draw [c,line width=0.9] (11.7649,1.51908) -- (11.7649,1.35632);
\draw [c,line width=0.9] (12.3792,1.51908) -- (12.3792,1.35632);
\draw [c,line width=0.9] (12.9935,1.68184) -- (12.9935,1.35632);
\draw [c,line width=0.9] (13.6078,1.51908) -- (13.6078,1.35632);
\draw [c,line width=0.9] (14.222,1.51908) -- (14.222,1.35632);
\draw [c,line width=0.9] (14.8363,1.51908) -- (14.8363,1.35632);
\draw [c,line width=0.9] (15.4506,1.51908) -- (15.4506,1.35632);
\draw [c,line width=0.9] (16.0649,1.68184) -- (16.0649,1.35632);
\draw [c,line width=0.9] (3.7792,1.68184) -- (3.7792,1.35632);
\draw [c,line width=0.9] (3.16491,1.51908) -- (3.16491,1.35632);
\draw [c,line width=0.9] (2.55063,1.51908) -- (2.55063,1.35632);
\draw [c,line width=0.9] (16.0649,1.68184) -- (16.0649,1.35632);
\draw [c,line width=0.9] (16.6792,1.51908) -- (16.6792,1.35632);
\draw [c,line width=0.9] (17.2935,1.51908) -- (17.2935,1.35632);
\draw [c,line width=0.9] (17.9077,1.51908) -- (17.9077,1.35632);
\draw [anchor=base] (3.7792,0.908736) node[scale=1.08496, color=c, rotate=0]{-2};
\draw [anchor=base] (6.85062,0.908736) node[scale=1.08496, color=c, rotate=0]{-1};
\draw [anchor=base] (9.92204,0.908736) node[scale=1.08496, color=c, rotate=0]{0};
\draw [anchor=base] (12.9935,0.908736) node[scale=1.08496, color=c, rotate=0]{1};
\draw [anchor=base] (16.0649,0.908736) node[scale=1.08496, color=c, rotate=0]{2};
\draw [c,line width=0.9] (2,1.35632) -- (2,12.2069);
\draw [anchor= east] (0.88,12.2069) node[scale=1.08496, color=c, rotate=90]{$V_{in} [V]$};
\draw [c,line width=0.9] (2.48,2.19282) -- (2,2.19282);
\draw [c,line width=0.9] (2.24,2.49365) -- (2,2.49365);
\draw [c,line width=0.9] (2.24,2.79447) -- (2,2.79447);
\draw [c,line width=0.9] (2.24,3.0953) -- (2,3.0953);
\draw [c,line width=0.9] (2.24,3.39613) -- (2,3.39613);
\draw [c,line width=0.9] (2.48,3.69696) -- (2,3.69696);
\draw [c,line width=0.9] (2.24,3.99779) -- (2,3.99779);
\draw [c,line width=0.9] (2.24,4.29861) -- (2,4.29861);
\draw [c,line width=0.9] (2.24,4.59944) -- (2,4.59944);
\draw [c,line width=0.9] (2.24,4.90027) -- (2,4.90027);
\draw [c,line width=0.9] (2.48,5.2011) -- (2,5.2011);
\draw [c,line width=0.9] (2.24,5.50193) -- (2,5.50193);
\draw [c,line width=0.9] (2.24,5.80275) -- (2,5.80275);
\draw [c,line width=0.9] (2.24,6.10358) -- (2,6.10358);
\draw [c,line width=0.9] (2.24,6.40441) -- (2,6.40441);
\draw [c,line width=0.9] (2.48,6.70524) -- (2,6.70524);
\draw [c,line width=0.9] (2.24,7.00607) -- (2,7.00607);
\draw [c,line width=0.9] (2.24,7.30689) -- (2,7.30689);
\draw [c,line width=0.9] (2.24,7.60772) -- (2,7.60772);
\draw [c,line width=0.9] (2.24,7.90855) -- (2,7.90855);
\draw [c,line width=0.9] (2.48,8.20938) -- (2,8.20938);
\draw [c,line width=0.9] (2.24,8.51021) -- (2,8.51021);
\draw [c,line width=0.9] (2.24,8.81104) -- (2,8.81104);
\draw [c,line width=0.9] (2.24,9.11186) -- (2,9.11186);
\draw [c,line width=0.9] (2.24,9.41269) -- (2,9.41269);
\draw [c,line width=0.9] (2.48,9.71352) -- (2,9.71352);
\draw [c,line width=0.9] (2.24,10.0143) -- (2,10.0143);
\draw [c,line width=0.9] (2.24,10.3152) -- (2,10.3152);
\draw [c,line width=0.9] (2.24,10.616) -- (2,10.616);
\draw [c,line width=0.9] (2.24,10.9168) -- (2,10.9168);
\draw [c,line width=0.9] (2.48,11.2177) -- (2,11.2177);
\draw [c,line width=0.9] (2.48,2.19282) -- (2,2.19282);
\draw [c,line width=0.9] (2.24,1.89199) -- (2,1.89199);
\draw [c,line width=0.9] (2.24,1.59116) -- (2,1.59116);
\draw [c,line width=0.9] (2.48,11.2177) -- (2,11.2177);
\draw [c,line width=0.9] (2.24,11.5185) -- (2,11.5185);
\draw [c,line width=0.9] (2.24,11.8193) -- (2,11.8193);
\draw [c,line width=0.9] (2.24,12.1201) -- (2,12.1201);
\draw [anchor= east] (1.9,2.19282) node[scale=1.08496, color=c, rotate=0]{-15};
\draw [anchor= east] (1.9,3.69696) node[scale=1.08496, color=c, rotate=0]{-10};
\draw [anchor= east] (1.9,5.2011) node[scale=1.08496, color=c, rotate=0]{-5};
\draw [anchor= east] (1.9,6.70524) node[scale=1.08496, color=c, rotate=0]{0};
\draw [anchor= east] (1.9,8.20938) node[scale=1.08496, color=c, rotate=0]{5};
\draw [anchor= east] (1.9,9.71352) node[scale=1.08496, color=c, rotate=0]{10};
\draw [anchor= east] (1.9,11.2177) node[scale=1.08496, color=c, rotate=0]{15};
\definecolor{c}{rgb}{0,0,1};
\foreach \P in {(13.2392,9.9241), (6.72776,3.48638), (10.2538,7.02712), (9.5934,6.38335), (11.2489,8.00482), (8.61055,5.4207), (12.2072,8.9434), (7.63691,4.46708), (14.1913,10.8867), (5.68348,2.52373), (15.2049,11.1876), (4.70063,2.37331),
 (16.1877,11.1274), (3.80991,2.43348), (16.4949,11.1876), (3.50277,2.43348)}{\draw[mark options={color=c,fill=c},mark size=1.681682pt,mark=*,mark size=1pt] plot coordinates {\P};}
\draw [c,line width=0.9] (5.36098,2.22861) -- (5.45312,2.319) -- (5.54527,2.4094) -- (5.63741,2.4998) -- (5.72955,2.59019) -- (5.82169,2.68059) -- (5.91384,2.77099) -- (6.00598,2.86138) -- (6.09812,2.95178) -- (6.19027,3.04218) -- (6.28241,3.13257)
 -- (6.37455,3.22297) -- (6.46669,3.31337) -- (6.55884,3.40376) -- (6.65098,3.49416) -- (6.74312,3.58456) -- (6.83526,3.67495) -- (6.92741,3.76535) -- (7.01955,3.85575) -- (7.11169,3.94614) -- (7.20384,4.03654) -- (7.29598,4.12694) --
 (7.38812,4.21733) -- (7.48026,4.30773) -- (7.57241,4.39812) -- (7.66455,4.48852) -- (7.75669,4.57892) -- (7.84883,4.66931) -- (7.94098,4.75971) -- (8.03312,4.85011) -- (8.12526,4.9405) -- (8.21741,5.0309) -- (8.30955,5.1213) -- (8.40169,5.21169) --
 (8.49383,5.30209) -- (8.58598,5.39249) -- (8.67812,5.48288) -- (8.77026,5.57328) -- (8.8624,5.66368) -- (8.95455,5.75407) -- (9.04669,5.84447) -- (9.13883,5.93487) -- (9.23098,6.02526) -- (9.32312,6.11566) -- (9.41526,6.20606) -- (9.5074,6.29645) --
 (9.59955,6.38685) -- (9.69169,6.47725) -- (9.78383,6.56764) -- (9.87597,6.65804);
\draw [c,line width=0.9] (9.87597,6.65804) -- (9.96812,6.74844) -- (10.0603,6.83883) -- (10.1524,6.92923) -- (10.2445,7.01963) -- (10.3367,7.11002) -- (10.4288,7.20042) -- (10.521,7.29082) -- (10.6131,7.38121) -- (10.7053,7.47161) --
 (10.7974,7.56201) -- (10.8895,7.6524) -- (10.9817,7.7428) -- (11.0738,7.8332) -- (11.166,7.92359) -- (11.2581,8.01399) -- (11.3503,8.10439) -- (11.4424,8.19478) -- (11.5345,8.28518) -- (11.6267,8.37558) -- (11.7188,8.46597) -- (11.811,8.55637) --
 (11.9031,8.64677) -- (11.9953,8.73716) -- (12.0874,8.82756) -- (12.1795,8.91796) -- (12.2717,9.00835) -- (12.3638,9.09875) -- (12.456,9.18915) -- (12.5481,9.27954) -- (12.6403,9.36994) -- (12.7324,9.46034) -- (12.8245,9.55073) -- (12.9167,9.64113)
 -- (13.0088,9.73153) -- (13.101,9.82192) -- (13.1931,9.91232) -- (13.2853,10.0027) -- (13.3774,10.0931) -- (13.4695,10.1835) -- (13.5617,10.2739) -- (13.6538,10.3643) -- (13.746,10.4547) -- (13.8381,10.5451) -- (13.9303,10.6355) -- (14.0224,10.7259)
 -- (14.1145,10.8163) -- (14.2067,10.9067) -- (14.2988,10.9971) -- (14.391,11.0875);
\draw [c,line width=0.9] (14.391,11.0875) -- (14.4831,11.1779);
\definecolor{c}{rgb}{1,0,0};
\draw [c,line width=0.9] (13.2392,9.9241) -- (13.1528,9.9241);
\draw [c,line width=0.9] (13.1528,9.86663) -- (13.1528,9.98157);
\draw [c,line width=0.9] (13.2392,9.9241) -- (13.3255,9.9241);
\draw [c,line width=0.9] (13.3255,9.86663) -- (13.3255,9.98157);
\draw [c,line width=0.9] (13.2392,9.9241) -- (13.2392,10.0082);
\draw [c,line width=0.9] (13.1817,10.0082) -- (13.2967,10.0082);
\draw [c,line width=0.9] (13.2392,9.9241) -- (13.2392,9.83998);
\draw [c,line width=0.9] (13.1817,9.83998) -- (13.2967,9.83998);
\draw [c,line width=0.9] (6.72776,3.48638) -- (6.64328,3.48638);
\draw [c,line width=0.9] (6.64328,3.42891) -- (6.64328,3.54385);
\draw [c,line width=0.9] (6.72776,3.48638) -- (6.81225,3.48638);
\draw [c,line width=0.9] (6.81225,3.42891) -- (6.81225,3.54385);
\draw [c,line width=0.9] (6.72776,3.48638) -- (6.72776,3.5705);
\draw [c,line width=0.9] (6.67029,3.5705) -- (6.78524,3.5705);
\draw [c,line width=0.9] (6.72776,3.48638) -- (6.72776,3.40226);
\draw [c,line width=0.9] (6.67029,3.40226) -- (6.78524,3.40226);
\draw [c,line width=0.9] (10.2538,7.02712) -- (10.2451,7.02712);
\draw [c,line width=0.9] (10.2451,6.96965) -- (10.2451,7.0846);
\draw [c,line width=0.9] (10.2538,7.02712) -- (10.2624,7.02712);
\draw [c,line width=0.9] (10.2624,6.96965) -- (10.2624,7.0846);
\draw [c,line width=0.9] (10.2538,7.02712) -- (10.2538,7.03554);
\draw [c,line width=0.9] (10.1963,7.03554) -- (10.3112,7.03554);
\draw [c,line width=0.9] (10.2538,7.02712) -- (10.2538,7.01871);
\draw [c,line width=0.9] (10.1963,7.01871) -- (10.3112,7.01871);
\draw [c,line width=0.9] (9.5934,6.38335) -- (9.58481,6.38335);
\draw [c,line width=0.9] (9.58481,6.32588) -- (9.58481,6.44082);
\draw [c,line width=0.9] (9.5934,6.38335) -- (9.60199,6.38335);
\draw [c,line width=0.9] (9.60199,6.32588) -- (9.60199,6.44082);
\draw [c,line width=0.9] (9.5934,6.38335) -- (9.5934,6.39176);
\draw [c,line width=0.9] (9.53593,6.39176) -- (9.65087,6.39176);
\draw [c,line width=0.9] (9.5934,6.38335) -- (9.5934,6.37494);
\draw [c,line width=0.9] (9.53593,6.37494) -- (9.65087,6.37494);
\draw [c,line width=0.9] (11.2489,8.00482) -- (11.2144,8.00482);
\draw [c,line width=0.9] (11.2144,7.94734) -- (11.2144,8.06229);
\draw [c,line width=0.9] (11.2489,8.00482) -- (11.2834,8.00482);
\draw [c,line width=0.9] (11.2834,7.94734) -- (11.2834,8.06229);
\draw [c,line width=0.9] (11.2489,8.00482) -- (11.2489,8.03865);
\draw [c,line width=0.9] (11.1914,8.03865) -- (11.3064,8.03865);
\draw [c,line width=0.9] (11.2489,8.00482) -- (11.2489,7.97098);
\draw [c,line width=0.9] (11.1914,7.97098) -- (11.3064,7.97098);
\draw [c,line width=0.9] (8.61055,5.4207) -- (8.57624,5.4207);
\draw [c,line width=0.9] (8.57624,5.36323) -- (8.57624,5.47817);
\draw [c,line width=0.9] (8.61055,5.4207) -- (8.64486,5.4207);
\draw [c,line width=0.9] (8.64486,5.36323) -- (8.64486,5.47817);
\draw [c,line width=0.9] (8.61055,5.4207) -- (8.61055,5.45431);
\draw [c,line width=0.9] (8.55308,5.45431) -- (8.66802,5.45431);
\draw [c,line width=0.9] (8.61055,5.4207) -- (8.61055,5.3871);
\draw [c,line width=0.9] (8.55308,5.3871) -- (8.66802,5.3871);
\draw [c,line width=0.9] (12.2072,8.9434) -- (12.1485,8.9434);
\draw [c,line width=0.9] (12.1485,8.88593) -- (12.1485,9.00087);
\draw [c,line width=0.9] (12.2072,8.9434) -- (12.2659,8.9434);
\draw [c,line width=0.9] (12.2659,8.88593) -- (12.2659,9.00087);
\draw [c,line width=0.9] (12.2072,8.9434) -- (12.2072,9.00091);
\draw [c,line width=0.9] (12.1497,9.00091) -- (12.2647,9.00091);
\draw [c,line width=0.9] (12.2072,8.9434) -- (12.2072,8.88589);
\draw [c,line width=0.9] (12.1497,8.88589) -- (12.2647,8.88589);
\draw [c,line width=0.9] (7.63691,4.46708) -- (7.57819,4.46708);
\draw [c,line width=0.9] (7.57819,4.40961) -- (7.57819,4.52455);
\draw [c,line width=0.9] (7.63691,4.46708) -- (7.69562,4.46708);
\draw [c,line width=0.9] (7.69562,4.40961) -- (7.69562,4.52455);
\draw [c,line width=0.9] (7.63691,4.46708) -- (7.63691,4.52458);
\draw [c,line width=0.9] (7.57943,4.52458) -- (7.69438,4.52458);
\draw [c,line width=0.9] (7.63691,4.46708) -- (7.63691,4.40957);
\draw [c,line width=0.9] (7.57943,4.40957) -- (7.69438,4.40957);
\draw [c,line width=0.9] (14.1913,10.8867) -- (14.0785,10.8867);
\draw [c,line width=0.9] (14.0785,10.8293) -- (14.0785,10.9442);
\draw [c,line width=0.9] (14.1913,10.8867) -- (14.3041,10.8867);
\draw [c,line width=0.9] (14.3041,10.8293) -- (14.3041,10.9442);
\draw [c,line width=0.9] (14.1913,10.8867) -- (14.1913,10.9972);
\draw [c,line width=0.9] (14.1339,10.9972) -- (14.2488,10.9972);
\draw [c,line width=0.9] (14.1913,10.8867) -- (14.1913,10.7763);
\draw [c,line width=0.9] (14.1339,10.7763) -- (14.2488,10.7763);
\draw [c,line width=0.9] (5.68348,2.52373) -- (5.57114,2.52373);
\draw [c,line width=0.9] (5.57114,2.46626) -- (5.57114,2.5812);
\draw [c,line width=0.9] (5.68348,2.52373) -- (5.79582,2.52373);
\draw [c,line width=0.9] (5.79582,2.46626) -- (5.79582,2.5812);
\draw [c,line width=0.9] (5.68348,2.52373) -- (5.68348,2.63421);
\draw [c,line width=0.9] (5.62601,2.63421) -- (5.74095,2.63421);
\draw [c,line width=0.9] (5.68348,2.52373) -- (5.68348,2.41324);
\draw [c,line width=0.9] (5.62601,2.41324) -- (5.74095,2.41324);
\draw [c,line width=0.9] (15.2049,11.1876) -- (15.0647,11.1876);
\draw [c,line width=0.9] (15.0647,11.1301) -- (15.0647,11.245);
\draw [c,line width=0.9] (15.2049,11.1876) -- (15.3451,11.1876);
\draw [c,line width=0.9] (15.3451,11.1301) -- (15.3451,11.245);
\draw [c,line width=0.9] (15.2049,11.1876) -- (15.2049,11.3027);
\draw [c,line width=0.9] (15.1474,11.3027) -- (15.2624,11.3027);
\draw [c,line width=0.9] (15.2049,11.1876) -- (15.2049,11.0725);
\draw [c,line width=0.9] (15.1474,11.0725) -- (15.2624,11.0725);
\draw [c,line width=0.9] (4.70063,2.37331) -- (4.56136,2.37331);
\draw [c,line width=0.9] (4.56136,2.31584) -- (4.56136,2.43079);
\draw [c,line width=0.9] (4.70063,2.37331) -- (4.83989,2.37331);
\draw [c,line width=0.9] (4.83989,2.31584) -- (4.83989,2.43079);
\draw [c,line width=0.9] (4.70063,2.37331) -- (4.70063,2.48609);
\draw [c,line width=0.9] (4.64315,2.48609) -- (4.7581,2.48609);
\draw [c,line width=0.9] (4.70063,2.37331) -- (4.70063,2.26054);
\draw [c,line width=0.9] (4.64315,2.26054) -- (4.7581,2.26054);
\draw [c,line width=0.9] (16.1877,11.1274) -- (16.0206,11.1274);
\draw [c,line width=0.9] (16.0206,11.0699) -- (16.0206,11.1849);
\draw [c,line width=0.9] (16.1877,11.1274) -- (16.3549,11.1274);
\draw [c,line width=0.9] (16.3549,11.0699) -- (16.3549,11.1849);
\draw [c,line width=0.9] (16.1877,11.1274) -- (16.1877,11.2416);
\draw [c,line width=0.9] (16.1303,11.2416) -- (16.2452,11.2416);
\draw [c,line width=0.9] (16.1877,11.1274) -- (16.1877,11.0132);
\draw [c,line width=0.9] (16.1303,11.0132) -- (16.2452,11.0132);
\draw [c,line width=0.9] (3.80991,2.43348) -- (3.64508,2.43348);
\draw [c,line width=0.9] (3.64508,2.37601) -- (3.64508,2.49095);
\draw [c,line width=0.9] (3.80991,2.43348) -- (3.97474,2.43348);
\draw [c,line width=0.9] (3.97474,2.37601) -- (3.97474,2.49095);
\draw [c,line width=0.9] (3.80991,2.43348) -- (3.80991,2.54534);
\draw [c,line width=0.9] (3.75244,2.54534) -- (3.86738,2.54534);
\draw [c,line width=0.9] (3.80991,2.43348) -- (3.80991,2.32162);
\draw [c,line width=0.9] (3.75244,2.32162) -- (3.86738,2.32162);
\draw [c,line width=0.9] (16.4949,11.1876) -- (16.3231,11.1876);
\draw [c,line width=0.9] (16.3231,11.1301) -- (16.3231,11.245);
\draw [c,line width=0.9] (16.4949,11.1876) -- (16.6667,11.1876);
\draw [c,line width=0.9] (16.6667,11.1301) -- (16.6667,11.245);
\draw [c,line width=0.9] (16.4949,11.1876) -- (16.4949,11.3027);
\draw [c,line width=0.9] (16.4374,11.3027) -- (16.5524,11.3027);
\draw [c,line width=0.9] (16.4949,11.1876) -- (16.4949,11.0725);
\draw [c,line width=0.9] (16.4374,11.0725) -- (16.5524,11.0725);
\draw [c,line width=0.9] (3.50277,2.43348) -- (3.33333,2.43348);
\draw [c,line width=0.9] (3.33333,2.37601) -- (3.33333,2.49095);
\draw [c,line width=0.9] (3.50277,2.43348) -- (3.67221,2.43348);
\draw [c,line width=0.9] (3.67221,2.37601) -- (3.67221,2.49095);
\draw [c,line width=0.9] (3.50277,2.43348) -- (3.50277,2.54534);
\draw [c,line width=0.9] (3.4453,2.54534) -- (3.56024,2.54534);
\draw [c,line width=0.9] (3.50277,2.43348) -- (3.50277,2.32162);
\draw [c,line width=0.9] (3.4453,2.32162) -- (3.56024,2.32162);
\definecolor{c}{rgb}{1,1,1};
\draw [color=c, fill=c] (16,10.8506) rectangle (18,12.2069);
\definecolor{c}{rgb}{0,0,0};
\draw [c,line width=0.9] (16,10.8506) -- (18,10.8506);
\draw [c,line width=0.9] (18,10.8506) -- (18,12.2069);
\draw [c,line width=0.9] (18,12.2069) -- (16,12.2069);
\draw [c,line width=0.9] (16,12.2069) -- (16,10.8506);
\draw [anchor=base west] (16.5,11.7152) node[scale=1.02114, color=c, rotate=0]{Data};
\definecolor{c}{rgb}{1,0,0};
\draw [c,line width=0.9] (16.075,11.8678) -- (16.425,11.8678);
\definecolor{c}{rgb}{0,0,1};
\foreach \P in {(16.25,11.8678)}{\draw[mark options={color=c,fill=c},mark size=1.681682pt,mark=*,mark size=1pt] plot coordinates {\P};}
\definecolor{c}{rgb}{0,0,0};
\draw [anchor=base west] (16.5,11.0371) node[scale=1.02114, color=c, rotate=0]{Fit};
\definecolor{c}{rgb}{0,0,1};
\draw [c,line width=0.9] (16.075,11.1897) -- (16.425,11.1897);
\definecolor{c}{rgb}{0,0,0};
\draw (10,13.1224) node[scale=1.40406, color=c, rotate=0]{Gain};
\end{tikzpicture}

 }%
 \caption{Curva di trasferimento di un amplificatore invertente} 
 \label{gr:amp_noninv.tex} 
\end{grafico}

In seguito sono presentati i dati del  \autoref{gr:amp_noninv.tex} acquisiti in laboratorio, con i rispettivi errori:

\begin{tabella}
 \centering
 \begin{center}
\begin{tabulary}{\textwidth}{CCCCCC}
\toprule
$V_{in+}$ $\pm$ $\sigma_{V_{in+}}$ (V) &$V_{in-}$ $\pm$ $\sigma_{V_{in-}}$(V) & FS (V) & $V_{out+}$ $\pm$ $\sigma_{V_{out+}}(V)$ (V) & $V_{out-}$ $\pm$ $\sigma_{V_{out-}}$(V) & FS (V) \\ \midrule 
1.08$\pm$ 0.03 & -1.04$\pm$ 0.03 & 0.3 & 10.7$\pm$ 0.3 & -10.7$\pm$ 0.3 & 3 \\ \midrule
0.108$\pm$ 0.003 & -0.107$\pm$ 0.003 & 0.03 & 1.07$\pm$ 0.003 & -1.07$\pm$ 0.03 & 0.3 \\ \midrule
0.43$\pm$ 0.01 & -0.43$\pm$ 0.01 & 0.120 & 4.3$\pm$ 0.1 & -4.3$\pm$ 0.1 & 1.2 \\ \midrule
0.74$\pm$ 0.02 & -0.74$\pm$ 0.02 & 0.2 & 7.4$\pm$ 0.2 & -7.4$\pm$ 0.2 & 2 \\ \midrule 
1.39$\pm$ 0.04 & -1.38$\pm$ 0.04 & 0.4 & 13.9$\pm$ 0.4 & -13.9$\pm$ 0.4 & 4 \\ \midrule
1.72$\pm$ 0.05 & -1.70$\pm$ 0.05 & 0.5 & 14.9$\pm$ 0.4 & -14.4$\pm$ 0.4 & 4 \\ \midrule
2.04$\pm$ 0.05 & -1.99$\pm$ 0.05 & 0.6 & 14.7$\pm$ 0.4 & -14.2$\pm$ 0.4 & 4 \\ \midrule
2.14$\pm$ 0.06 & -2.09$\pm$ 0.06 & 0.6 & 14.9$\pm$ 0.4 & -14.2$\pm$ 0.4 & 4 \\ \midrule


\bottomrule
\end{tabulary}
\end{center}

 \caption{Dati curva di trasferimento}
 \label{tab:tab_non_inv.tex}
\end{tabella}
E' stata fatta l'interpolazione lineare pesata dei punti nel  \autoref{gr:amp_noninv.tex} compresi tra 0 e 1.5 V.\\
$q = -0.007 \pm 0.03 \, V$\\
$m = 10.0 \pm 0.1 \,\frac{V}{V}$


